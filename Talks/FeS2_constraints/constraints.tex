% see https://tex.stackexchange.com/a/74828
\PassOptionsToPackage{x11names}{xcolor}
%\documentclass[10pt, xcolor=x11names, compress]{beamer}
\documentclass[10pt, xcolor=x11names, compress]{beamer}

\usetheme{progressbar}
%\usecolortheme[named=Purple4]{structure}
\progressbaroptions{headline=sections,titlepage=normal,frametitle=normal}

\setbeamertemplate{navigation symbols}{}

\usepackage{iwona} 

\usepackage{alltt}
\usepackage{amsmath,amsfonts, amssymb, amscd}
\usepackage{hyperref}
\usepackage{setspace}
\usepackage{wasysym}
\usepackage{ulem}
\usepackage{xspace}

\usepackage{calc}
\usepackage[overlay,absolute]{textpos}
\TPGrid[5mm,5mm]{20}{20}


\usepackage[x11names]{xcolor}

\usepackage{marvosym}
\newcommand{\homepagesymbol}{{\Large\ComputerMouse~}}%      {{\Large\marvosymbol{205}}~}


\renewcommand{\Re}{\operatorname{Re}}
\renewcommand{\Im}{\operatorname{Im}}
\newcommand{\debye}{\operatorname{debye}}

\newcommand{\chik}{$\chi(k)$}
\newcommand{\chir}{$|\tilde{\chi}(R)|$}


\newcommand{\file}[1]{{\color{Firebrick4}\texttt{`#1'}}}
\newcommand{\multiple}{{\color{Orange3}\textsl{multiple}}}

\definecolor{Programs}{rgb}{0.0,0.2,0.0}
\newcommand{\atoms}    {{\color{Programs}\textsc{atoms}}\xspace}
\newcommand{\feff}     {{\color{Programs}\textsc{feff}}\xspace}
\newcommand{\feffex}   {{\color{Programs}\textsc{feff85exafs}}\xspace}
\newcommand{\feffsix}  {{\color{Programs}\textsc{feff6}}\xspace}
\newcommand{\feffeight}{{\color{Programs}\textsc{feff8}}\xspace}
\newcommand{\feffnine} {{\color{Programs}\textsc{feff9}}\xspace}
\newcommand{\ifeffit}  {{\color{Programs}\textsc{ifeffit}}\xspace}
\newcommand{\larch}    {{\color{Programs}\textsc{larch}}\xspace}
\newcommand{\athena}   {{\color{Programs}\textsc{athena}}\xspace}
\newcommand{\artemis}  {{\color{Programs}\textsc{artemis}}\xspace}

\renewenvironment<>{center}
{\begin{actionenv}#1\begin{originalcenter}}
{\end{originalcenter}\end{actionenv}}

\definecolor{guessp}   {rgb}{0.64,0.00,0.64}
\newcommand{\guessp}   {{\color{guessp}guess}}
\definecolor{defp}     {rgb}{0.00,0.55,0.00}
\newcommand{\defp}     {{\color{defp}def}}
\definecolor{setp}     {rgb}{0,0,0}
\newcommand{\setp}     {{\color{setp}set}}
\definecolor{lguessp}  {rgb}{0.24,0.11,0.56}
\newcommand{\lguessp}  {{\color{lguessp}lguess}}
\definecolor{skipp}    {rgb}{0.70,0.70,0.70}
\newcommand{\skipp}    {{\color{skipp}skip}}
\definecolor{restrainp}{rgb}{0.80,0.61,0.11}
\newcommand{\restrainp}{{\color{restrainp}restrain}}
\definecolor{afterp}   {rgb}{0.29,0.44,0.55}
\newcommand{\afterp}   {{\color{afterp}after}}
\definecolor{penaltyp} {rgb}{0.55,0.35,0.17}
\newcommand{\penaltyp} {{\color{penaltyp}penalty}}
\definecolor{mergep}   {rgb}{0.93,0.00,0.00}
\newcommand{\mergep}   {{\color{mergep}merge}}

\hyphenation{EXAFS}

\newcommand{\exafsequation}[1][\small]{%
  {#1
    \begin{align*}
      \chi(k,\Gamma) =& 
      { \frac{{\color{Red4}(N_\Gamma S_0^2)}{\color{Blue4}F_\Gamma(k)}
                        e^{-2{\color{Red4}\sigma_\Gamma^2}k^2}
                        e^{-2R_\Gamma/{\color{Blue4}\lambda(k)}}
                        }
                        {2\,kR_\Gamma^2} }
      \sin{(2kR_\Gamma + {\color{Blue4}\Phi_\Gamma(k)})} \\
      \chi_{\mathrm{theory}}(k) =& \sum\limits_{\Gamma}\chi(k,\Gamma)\\
      R_\Gamma =& \> {\color{Blue4}R_{0,\Gamma}} +
      {\color{Red4}\Delta R_\Gamma} \\
      k =& \sqrt{2m_e(E_0 - {\color{Red4}\Delta E_0})/\hbar^2} 
           \approx \sqrt{(E_0 - {\color{Red4}\Delta E_0})/3.81}
    \end{align*}}
}


%% inline enumeration, see
%% http://tex.stackexchange.com/questions/94478/beamer-inline-itemize-and-enumeration/94521#94521
\newcounter{newenumi}
\setcounter{newenumi}{1}
\newcommand{\inlineenum}{%
 {%
 \setcounter{enumi}{\thenewenumi}%
 \leavevmode\usebeamertemplate{enumerate  item}
 \stepcounter{newenumi}
 \setcounter{enumi}{0}
 }
}
\newcommand{\resetinlineenum}{\setcounter{newenumi}{1}}

\usepackage{xparse}
\definecolor{bngray}{rgb}{0.5,0.5,0.5}
\NewDocumentEnvironment{bottomnote}{O{0.5}O{19.5}}%[0.85][19.5]
{\begin{textblock*}{#1\linewidth}(0pt,#2\TPVertModule)%
   \tiny\begin{color}{bngray}}%
{\end{color}\end{textblock*}}

\NewDocumentCommand{\cornerlogo}{m}
{\begin{textblock*}{0.08\linewidth}(17.2\TPHorizModule,0\TPVertModule)%
    \includegraphics[width=2cm]{#1}%
  \end{textblock*}} 
\NewDocumentCommand{\smcornerlogo}{m}
{\begin{textblock*}{0.08\linewidth}(19.5\TPHorizModule,0\TPVertModule)%
    \includegraphics[width=7mm]{#1}%
  \end{textblock*}} 


\DeclareDocumentCommand{\doiref}{mO{Blue4}}%
{\href{https://doi.org/#1}{\color{#2}{\ComputerMouse~}DOI:~#1}}

\DeclareDocumentCommand{\inlinelogo}{m}%
{\raisebox{-.2\height}{\includegraphics[width=5mm]{#1}}\xspace}

\DeclareDocumentCommand{\titlepageurl}{mO{Blue4}}%
{~\\[2ex]{\footnotesize\href{#1}{\color{#2}%
    {\ComputerMouse\,}PDF of this talk: #1}}}
\usepackage{ulem}


\newcommand{\fes}{FeS$_2$}
\newtheorem{conclusion}[theorem]{Conclusion}
\newtheorem{assertion}[theorem]{Assertion}
\newtheorem{exercise}[theorem]{Exercise for the reader}
\newtheorem{remember}[theorem]{Always remember}

\mode<presentation>

\title{FeS$_2$ EXAFS}%
\subtitle{Talking about constraints}
\include{author}
\date{\today}

\begin{document}

\maketitle

\begin{frame}
  \frametitle{The EXAFS Equation}
  \exafsequation

  \smallskip

  \par\noindent\rule{\textwidth}{0.4pt}

  \begin{overlayarea}{\linewidth}{0.5\textheight} 
  \end{overlayarea}
\end{frame}

\begin{frame}
  \frametitle{The Problem}
  \exafsequation

  \smallskip

  \par\noindent\rule{\textwidth}{0.4pt}

  \begin{overlayarea}{\linewidth}{0.5\textheight} 
    \begin{enumerate}
    \item There are 5 parameters that must be determined for each
      path: {\color{Red4} N}, {\color{Red4} $S_0^2$},
      {\color{Red4} $\Delta E_0$}, {\color{Red4} $\Delta R$}, and
      {\color{Red4} $\sigma^2$}
    \item We are currently considering 4 paths
    \item That's 20 numbers, but \artemis tells us that we have
      fewer than 17 independent points.
    \end{enumerate}
    \begin{alertblock}{}
      \begin{center}
        Are we doomed?  Is EXAFS hopeless?
      \end{center}
    \end{alertblock}
  \end{overlayarea}
\end{frame}


\begin{frame}
  \frametitle{The Problem}
  \exafsequation

  \smallskip

  \par\noindent\rule{\textwidth}{0.4pt}
  
  \begin{overlayarea}{\linewidth}{0.5\textheight} 
    \begin{center}
      \begin{tabular}{ccccl}
        Npaths & Nparams & Constraints & count & comment \\
        \hline
        4 & 5 & 0 & 20 & too many!\\
        \hline
      \end{tabular}
    \end{center}
  \end{overlayarea}
\end{frame}

\begin{frame}
  \frametitle{Coordination number}
  \exafsequation

  \smallskip

  \par\noindent\rule{\textwidth}{0.4pt}
  
  \begin{overlayarea}{\linewidth}{0.5\textheight} 
    \begin{center}
      \begin{tabular}{ccccl}
        Npaths & Nparams & Constraints & count & comment \\
        \hline
        {\color{Gray0}4} & {\color{Gray0}5} & {\color{Gray0}0} & {\color{Gray0}20} & {\color{Gray0}too many!}\\
        4 & \alert{4} & \alert{0} & \alert{16} & \alert{use N from CIF file!}\\
        \hline
      \end{tabular}
    \end{center}
  \end{overlayarea}
\end{frame}

\begin{frame}
  \frametitle{$S_0^2$: Amplitude reduction factor}
  \exafsequation

  \smallskip

  \par\noindent\rule{\textwidth}{0.4pt}
  
  \begin{overlayarea}{\linewidth}{0.5\textheight} 
    \begin{center}
      \begin{tabular}{ccccl}
        Npaths & Nparams & Constraints & count & comment \\
        \hline
        {\color{Gray0}4} & {\color{Gray0}5} & {\color{Gray0}0} & {\color{Gray0}20} & {\color{Gray0}too many!}\\
        {\color{Gray0}4} & {\color{Gray0}4} & {\color{Gray0}0} & {\color{Gray0}16} & {\color{Gray0}use N from CIF file!}\\
        4 & \alert{3} & \alert{1} & \alert{13} & \alert{$S_0^2$ is common to all paths!}\\
        \hline
      \end{tabular}
    \end{center}
  \end{overlayarea}
\end{frame}

\begin{frame}
  \frametitle{$\Delta E_0$: Shift of the 0 of wavenumber}
  \exafsequation

  \smallskip

  \par\noindent\rule{\textwidth}{0.4pt}
  
  \begin{overlayarea}{\linewidth}{0.5\textheight} 
    \begin{center}
      \begin{tabular}{ccccl}
        Npaths & Nparams & Constraints & count & comment \\
        \hline
        {\color{Gray0}4} & {\color{Gray0}5} & {\color{Gray0}0} & {\color{Gray0}20} & {\color{Gray0}too many!}\\
        {\color{Gray0}4} & {\color{Gray0}4} & {\color{Gray0}0} & {\color{Gray0}16} & {\color{Gray0}use N from CIF file!}\\
        {\color{Gray0}4} & {\color{Gray0}3} & {\color{Gray0}1} & {\color{Gray0}13} & {\color{Gray0}$S_0^2$ is common to all paths!}\\
        4 & \alert{2} & \alert{2} & \alert{10} & \alert{$\Delta E_0$ is common to all paths!}\\
        \hline
      \end{tabular}
    \end{center}
  \end{overlayarea}
\end{frame}

\begin{frame}
  \frametitle{Isotropic lattice expansion/contraction}
  \exafsequation

  \smallskip

  \par\noindent\rule{\textwidth}{0.4pt}
  
  \begin{overlayarea}{\linewidth}{0.5\textheight} 
    \begin{center}
      \begin{tabular}{ccccl}
        Npaths & Nparams & Constraints & count & comment \\
        \hline
        {\color{Gray0}4} & {\color{Gray0}5} & {\color{Gray0}0} & {\color{Gray0}20} & {\color{Gray0}too many!}\\
        {\color{Gray0}4} & {\color{Gray0}4} & {\color{Gray0}0} & {\color{Gray0}16} & {\color{Gray0}use N from CIF file!}\\
        {\color{Gray0}4} & {\color{Gray0}3} & {\color{Gray0}1} & {\color{Gray0}13} & {\color{Gray0}$S_0^2$ is common to all paths!}\\
        {\color{Gray0}4} & {\color{Gray0}2} & {\color{Gray0}2} & {\color{Gray0}10} & {\color{Gray0}$\Delta S_0$ is common to all paths!}\\
        4 & \alert{1} & \alert{3} & \alert{7} & \alert{Compute $\Delta R$ from $\alpha$!}\\
        \hline
      \end{tabular}
    \end{center}
  \end{overlayarea}
\end{frame}

\begin{frame}
  \frametitle{Mean square deviations in R}
  \exafsequation

  \smallskip

  \par\noindent\rule{\textwidth}{0.4pt}
  
  \begin{overlayarea}{\linewidth}{0.5\textheight} 
    \begin{center}
      \begin{tabular}{ccccl}
        Npaths & Nparams & Constraints & count & comment \\
        \hline
        {\color{Gray0}4} & {\color{Gray0}5} & {\color{Gray0}0} & {\color{Gray0}20} & {\color{Gray0}too many!}\\
        {\color{Gray0}4} & {\color{Gray0}4} & {\color{Gray0}0} & {\color{Gray0}16} & {\color{Gray0}use N from CIF file!}\\
        {\color{Gray0}4} & {\color{Gray0}3} & {\color{Gray0}1} & {\color{Gray0}13} & {\color{Gray0}$S_0^2$ is common to all paths!}\\
        {\color{Gray0}4} & {\color{Gray0}2} & {\color{Gray0}2} & {\color{Gray0}10} & {\color{Gray0}$\Delta S_0$ is common to all paths!}\\
        4 & \alert{1} & \alert{3} & \alert{7} & \alert{Compute $\Delta R$ from $\alpha$!}\\
        \hline
      \end{tabular}
    \end{center}

    Sadly, there is no such constraint that is easy to figure out for
    $\sigma^2$, so will start with a $\sigma^2$ for each path.
    
  \end{overlayarea}
\end{frame}


\end{document}


%%% Local Variables:
%%% TeX-parse-self: t
%%% TeX-auto-save: t
%%% TeX-auto-untabify: t
%%% TeX-PDF-mode: t
%%% End:
