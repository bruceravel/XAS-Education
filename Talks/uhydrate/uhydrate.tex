% see https://tex.stackexchange.com/a/74828
\PassOptionsToPackage{x11names}{xcolor}
\documentclass[10pt, xcolor=x11names, compress]{beamer}
%\documentclass[10pt, xcolor=x11names, compress, handout]{beamer}
\usetheme{progressbar}
%\usecolortheme[named=Purple4]{structure}
\progressbaroptions{headline=sections,titlepage=normal,frametitle=normal}

\setbeamertemplate{navigation symbols}{}

\usepackage{iwona} 

\usepackage{alltt}
\usepackage{amsmath,amsfonts, amssymb, amscd}
\usepackage{hyperref}
\usepackage{setspace}
\usepackage{wasysym}
\usepackage{ulem}
\usepackage{xspace}

\usepackage{calc}
\usepackage[overlay,absolute]{textpos}
\TPGrid[5mm,5mm]{20}{20}


\usepackage[x11names]{xcolor}

\usepackage{marvosym}
\newcommand{\homepagesymbol}{{\Large\ComputerMouse~}}%      {{\Large\marvosymbol{205}}~}


\renewcommand{\Re}{\operatorname{Re}}
\renewcommand{\Im}{\operatorname{Im}}
\newcommand{\debye}{\operatorname{debye}}

\newcommand{\chik}{$\chi(k)$}
\newcommand{\chir}{$|\tilde{\chi}(R)|$}


\newcommand{\file}[1]{{\color{Firebrick4}\texttt{`#1'}}}
\newcommand{\multiple}{{\color{Orange3}\textsl{multiple}}}

\definecolor{Programs}{rgb}{0.0,0.2,0.0}
\newcommand{\atoms}    {{\color{Programs}\textsc{atoms}}\xspace}
\newcommand{\feff}     {{\color{Programs}\textsc{feff}}\xspace}
\newcommand{\feffex}   {{\color{Programs}\textsc{feff85exafs}}\xspace}
\newcommand{\feffsix}  {{\color{Programs}\textsc{feff6}}\xspace}
\newcommand{\feffeight}{{\color{Programs}\textsc{feff8}}\xspace}
\newcommand{\feffnine} {{\color{Programs}\textsc{feff9}}\xspace}
\newcommand{\ifeffit}  {{\color{Programs}\textsc{ifeffit}}\xspace}
\newcommand{\larch}    {{\color{Programs}\textsc{larch}}\xspace}
\newcommand{\athena}   {{\color{Programs}\textsc{athena}}\xspace}
\newcommand{\artemis}  {{\color{Programs}\textsc{artemis}}\xspace}

\renewenvironment<>{center}
{\begin{actionenv}#1\begin{originalcenter}}
{\end{originalcenter}\end{actionenv}}

\definecolor{guessp}   {rgb}{0.64,0.00,0.64}
\newcommand{\guessp}   {{\color{guessp}guess}}
\definecolor{defp}     {rgb}{0.00,0.55,0.00}
\newcommand{\defp}     {{\color{defp}def}}
\definecolor{setp}     {rgb}{0,0,0}
\newcommand{\setp}     {{\color{setp}set}}
\definecolor{lguessp}  {rgb}{0.24,0.11,0.56}
\newcommand{\lguessp}  {{\color{lguessp}lguess}}
\definecolor{skipp}    {rgb}{0.70,0.70,0.70}
\newcommand{\skipp}    {{\color{skipp}skip}}
\definecolor{restrainp}{rgb}{0.80,0.61,0.11}
\newcommand{\restrainp}{{\color{restrainp}restrain}}
\definecolor{afterp}   {rgb}{0.29,0.44,0.55}
\newcommand{\afterp}   {{\color{afterp}after}}
\definecolor{penaltyp} {rgb}{0.55,0.35,0.17}
\newcommand{\penaltyp} {{\color{penaltyp}penalty}}
\definecolor{mergep}   {rgb}{0.93,0.00,0.00}
\newcommand{\mergep}   {{\color{mergep}merge}}

\hyphenation{EXAFS}

\newcommand{\exafsequation}[1][\small]{%
  {#1
    \begin{align*}
      \chi(k,\Gamma) =& 
      { \frac{{\color{Red4}(N_\Gamma S_0^2)}{\color{Blue4}F_\Gamma(k)}
                        e^{-2{\color{Red4}\sigma_\Gamma^2}k^2}
                        e^{-2R_\Gamma/{\color{Blue4}\lambda(k)}}
                        }
                        {2\,kR_\Gamma^2} }
      \sin{(2kR_\Gamma + {\color{Blue4}\Phi_\Gamma(k)})} \\
      \chi_{\mathrm{theory}}(k) =& \sum\limits_{\Gamma}\chi(k,\Gamma)\\
      R_\Gamma =& \> {\color{Blue4}R_{0,\Gamma}} +
      {\color{Red4}\Delta R_\Gamma} \\
      k =& \sqrt{2m_e(E_0 - {\color{Red4}\Delta E_0})/\hbar^2} 
           \approx \sqrt{(E_0 - {\color{Red4}\Delta E_0})/3.81}
    \end{align*}}
}


%% inline enumeration, see
%% http://tex.stackexchange.com/questions/94478/beamer-inline-itemize-and-enumeration/94521#94521
\newcounter{newenumi}
\setcounter{newenumi}{1}
\newcommand{\inlineenum}{%
 {%
 \setcounter{enumi}{\thenewenumi}%
 \leavevmode\usebeamertemplate{enumerate  item}
 \stepcounter{newenumi}
 \setcounter{enumi}{0}
 }
}
\newcommand{\resetinlineenum}{\setcounter{newenumi}{1}}

\usepackage{xparse}
\definecolor{bngray}{rgb}{0.5,0.5,0.5}
\NewDocumentEnvironment{bottomnote}{O{0.5}O{19.5}}%[0.85][19.5]
{\begin{textblock*}{#1\linewidth}(0pt,#2\TPVertModule)%
   \tiny\begin{color}{bngray}}%
{\end{color}\end{textblock*}}

\NewDocumentCommand{\cornerlogo}{m}
{\begin{textblock*}{0.08\linewidth}(17.2\TPHorizModule,0\TPVertModule)%
    \includegraphics[width=2cm]{#1}%
  \end{textblock*}} 
\NewDocumentCommand{\smcornerlogo}{m}
{\begin{textblock*}{0.08\linewidth}(19.5\TPHorizModule,0\TPVertModule)%
    \includegraphics[width=7mm]{#1}%
  \end{textblock*}} 


\DeclareDocumentCommand{\doiref}{mO{Blue4}}%
{\href{https://doi.org/#1}{\color{#2}{\ComputerMouse~}DOI:~#1}}

\DeclareDocumentCommand{\inlinelogo}{m}%
{\raisebox{-.2\height}{\includegraphics[width=5mm]{#1}}\xspace}

\DeclareDocumentCommand{\titlepageurl}{mO{Blue4}}%
{~\\[2ex]{\footnotesize\href{#1}{\color{#2}%
    {\ComputerMouse\,}PDF of this talk: #1}}}

%% define new commands here
%\newcommand{\eto}{EuTiO$_3$}

\mode<presentation>

\title{Discussion of the Uranyl Hydrate EXAFS Analysis Example}
%\subtitle{}
\include{author}
\include{date}
%\date[Diamond2011]{EXAFS Data Analysis workshop 2011\\
%  Diamond Light Source\\November 14--17, 2011}

\begin{document}
\maketitle

\begin{frame}
  \frametitle{Copyright}
  \tiny

  This document is copyright \copyright\ 2010-2015 Bruce Ravel.

  \begin{center}
    \includegraphics[width=1.0cm]{cc-by-sa.png}
  \end{center}

  This work is licensed under the Creative Commons
  Attribution-ShareAlike License.  To view a copy of this license,
  visit \href{http://creativecommons.org/licenses/by-sa/3.0/}
  {\color{Purple2}\texttt{http://creativecommons.org/licenses/by-sa/3.0/}}
  or send a letter to Creative Commons, 559 Nathan Abbott Way,
  Stanford, California 94305, USA.

  \begin{description}[Under the following conditions:]
  \tiny
  \item[You are free:] %
    \begin{itemize}
      \tiny
    \item \textbf{to Share} --- to copy, distribute, and transmit the work
    \item \textbf{to Remix} --- to adapt the work
    \item to make commercial use of the work
    \end{itemize}
  \item[Under the following conditions:] %
    \begin{itemize}
      \tiny
    \item \textbf{Attribution} -- You must attribute the work in the manner
      specified by the author or licensor (but not in any way that
      suggests that they endorse you or your use of the work).
    \item \textbf{Share Alike} -- If you alter, transform, or build upon this
      work, you may distribute the resulting work only under the same,
      similar or a compatible license.
    \end{itemize}
  \item[With the understanidng that:] 
    \begin{itemize}
      \tiny
    \item \textbf{Waiver} -- Any of the above conditions can be waived
      if you get permission from the copyright holder.
    \item \textbf{Public Domain} -- Where the work or any of its
      elements is in the public domain under applicable law, that
      status is in no way affected by the license.
    \item \textbf{Other Rights} -- In no way are any of the following
      rights affected by the license:
      \begin{itemize}
      \tiny
      \item Your fair dealing or fair use rights, or other
        applicable copyright exceptions and limitations;
      \item The author's moral rights;
      \item Rights other persons may have either in the work itself
        or in how the work is used, such as publicity or privacy
        rights.
      \end{itemize}
    \item \textbf{Notice} -- For any reuse or distribution, you must
      make clear to others the license terms of this work.
    \end{itemize}
  \end{description}

  This is a human-readable summary of the Legal Code (the full
  license).


\end{frame}

%%% Local Variables:
%%% mode: latex
%%% End:


\begin{frame}
  \frametitle{The sample}
  \begin{columns}[T]
    \begin{column}{0.5\linewidth}
      1000 ppm uranyl nitrate solution dissolved in DI water, pH=0.96\\[3ex]

      \includegraphics[width=\linewidth]{uranyl.pdf}

      \bigskip

      The liquid was held in a fluid cell and measured in fluorescence.
    \end{column}
    \begin{column}{0.5\linewidth}
      \includegraphics[width=\linewidth]{images/uhydrate_mu.png}

      \includegraphics[width=\linewidth]{images/uhydrate_chi.png}
    \end{column}
  \end{columns}

  \begin{bottomnote}[0.7][19.0]
    S.D.\ Kelly, et al., \textit{X-ray absorption fine structure
      determination of pH-dependent U-bacterial cell wall
      interactions}, Geochimica et Cosmochimica Acta \textbf{66}:22
    (2002) pp\  3855-3871.
    \doiref{10.1016/S0016-7037(02)00947-X}[LightBlue4]
  \end{bottomnote}
\end{frame}

\begin{frame}
  \frametitle{Statistical noise and systematic noise}
  \begin{columns}[T]
    \begin{column}{0.5\linewidth}
      \includegraphics[width=\linewidth]{images/merged_chi.png}
      
    \end{column}
    \begin{column}{0.5\linewidth}
      \begin{tabular}{lr}
        data & \multicolumn{1}{c}{$\epsilon_k$} \\
        \hline
        5 scans & $\sim4.76\times10^{-3}$ \\
        merge   & $1.77\times10^{-3}$
      \end{tabular}
      $$\frac{\epsilon_1}{\epsilon_{merge}} = 2.69$$
      $$\sqrt{5} \approx 2.24$$
    \end{column}
  \end{columns}
  \begin{block}{The data seem show the behavior of statistcal noise}
    But is the signal at 14\,\AA$^{-1}$ really data?  In fact, is the
    signal beyond 9\,\AA$^{-1}$ really data?
  \end{block}
  \begin{alertblock}{}
    \centering In any case, how to we proceed with a molecule in solution?
  \end{alertblock}
\end{frame}

\begin{frame}[fragile]
  \frametitle{Sodium uranyl triacetate}
  \begin{columns}[T]
    \begin{column}{0.7\linewidth}
      Here's a crystal that contains the uranyl moiety:
      \begin{center}
        \begin{minipage}{0.75\linewidth}
          \begin{block}{}
            \tiny
            \begin{alltt}
{\color{Green4}title Na uranyl triacetate}
{\color{Green4}title Templeton, et al, Acta Cryst 1985 C41 1439-1441}
{\color{SteelBlue4}space} = P 21 3
{\color{SteelBlue4}rmax}  = 7.0    {\color{SteelBlue4}a}=10.689
{\color{SteelBlue4}core}  = U
{\color{Purple4}atoms}
{\color{Blue4}! At.type   x         y         z      tag}
  U        0.4294    0.4294    0.4294  U 
  Na       0.8286    0.8286    0.8286  Na
  O        0.3343    0.3343    0.3343  Oax
  O        0.5242    0.5242    0.5242  Oax
  O        0.3834    0.2945    0.6110  Oeq
  O        0.5464    0.2443    0.5007  Oeq
  C        0.4786    0.2260    0.5950  C
  C        0.5088    0.1240    0.6862  C 
            \end{alltt}
          \end{block}
        \end{minipage}
      \end{center}
      If we ignore the Na and C, this has all the scatterers we need
      at the approximate distances we need.
    \end{column}
    \begin{column}{0.3\linewidth}
      \includegraphics[width=\linewidth]{../ATEA/mfc/NaU_triacetate_full.png}

      \includegraphics[width=\linewidth]{../ATEA/mfc/NaU_triacetate.png}
    \end{column}
  \end{columns}
\end{frame}

\defverbatim{\StyleElem}{%
\tiny
\begin{verbatim}
     0     92     U         
     1     92     U         
     2     11     Na        
     3     8      O         
     4     6      C
\end{verbatim}
\vspace{-3ex}
}
\defverbatim{\StyleSites}{%
\tiny
\begin{verbatim}
     0     92     U         
     1     92     U         
     2     11     Na        
     3     8      Oax         
     4     8      Oax         
     5     8      Oeq         
     6     8      Oeq         
     7     6      C
     8     6      C
\end{verbatim}
\vspace{-3ex}
}
\defverbatim{\StyleTags}{%
\tiny
\begin{verbatim}
     0     92     U
     1     92     U
     2     11     Na
     3     8      Oax
     4     8      Oeq
     5     6      C
\end{verbatim}
\vspace{-3ex}
}


\defverbatim[colored]{\AtomsListElem}{%
\tiny
\begin{alltt}
 {\color{SteelBlue4}ATOMS}
 {\color{Blue4}* x          y          z     ipot tag      distance}
  0.00000    0.00000    0.00000  0  U        0.00000
  1.01332    1.01332    1.01332  3  Oax.1    1.75512
 -1.01652   -1.01652   -1.01652  3  Oax.2    1.76067
  1.25061   -1.97853    0.76213  3  Oeq.1    2.46160
 -1.97853    0.76213    1.25061  3  Oeq.1    2.46160
  0.76213    1.25061   -1.97853  3  Oeq.1    2.46160
 -0.49169   -1.44195    1.94112  3  Oeq.2    2.46758
 -1.44195    1.94112   -0.49169  3  Oeq.2    2.46758
  1.94112   -0.49169   -1.44195  3  Oeq.2    2.46758
\end{alltt}
}

\defverbatim[colored]{\AtomsListSites}{%
\tiny
\begin{alltt}
 {\color{SteelBlue4}ATOMS}
 {\color{Blue4}* x          y          z     ipot tag      distance}
  0.00000    0.00000    0.00000  0  U        0.00000
  1.01332    1.01332    1.01332  4  Oax.1    1.75512
 -1.01652   -1.01652   -1.01652  3  Oax.2    1.76067
  1.25061   -1.97853    0.76213  6  Oeq.1    2.46160
 -1.97853    0.76213    1.25061  6  Oeq.1    2.46160
  0.76213    1.25061   -1.97853  6  Oeq.1    2.46160
 -0.49169   -1.44195    1.94112  5  Oeq.2    2.46758
 -1.44195    1.94112   -0.49169  5  Oeq.2    2.46758
  1.94112   -0.49169   -1.44195  5  Oeq.2    2.46758
\end{alltt}
}

\defverbatim[colored]{\AtomsListTags}{%
\tiny
\begin{alltt}
 {\color{SteelBlue4}ATOMS}
 {\color{Blue4}* x          y          z     ipot tag      distance}
  0.00000    0.00000    0.00000  0  U        0.00000
 {\color{DeepPink2} 1.01332    1.01332    1.01332  3  Oax.1    1.75512}
 {\color{DeepPink2}-1.01652   -1.01652   -1.01652  3  Oax.2    1.76067}
  1.25061   -1.97853    0.76213  4  Oeq.1    2.46160
 -1.97853    0.76213    1.25061  4  Oeq.1    2.46160
  0.76213    1.25061   -1.97853  4  Oeq.1    2.46160
 -0.49169   -1.44195    1.94112  4  Oeq.2    2.46758
 -1.44195    1.94112   -0.49169  4  Oeq.2    2.46758
  1.94112   -0.49169   -1.44195  4  Oeq.2    2.46758
\end{alltt}
}
  



\begin{frame}[fragile]
  \frametitle{Unique potentials in Feff}
  \begin{columns}[T]
    \begin{column}{0.5\linewidth}
      \includegraphics[width=0.9\linewidth]{images/atoms.png}
    \end{column}
    \begin{column}{0.5\linewidth}
      Style = \only<1>{elem}\only<2-3>{sites}\only<4>{tags}\\
      \vspace{-2ex}
      \begin{block}{}
        \tiny\begin{alltt}
 {\color{SteelBlue4}POTENTIALS}
  {\color{Blue4}* ipot   Z      tag}
        \end{alltt}
        \vspace{-9ex}
        \only<1>{\StyleElem}\only<2-3>{\StyleSites}\only<4>{\StyleTags}
      \end{block}
      \only<1>{
        \begin{exampleblock}{}
          \AtomsListElem
        \end{exampleblock}
      }
      \only<2>{
        \begin{exampleblock}{}
          \AtomsListSites
        \end{exampleblock}
      }
      \only<3>{\includegraphics[width=\linewidth]{images/sites_error.png}}
      \only<4>{
        \begin{exampleblock}{}
          \AtomsListTags
        \end{exampleblock}
      }
    \end{column}
  \end{columns}
  \begin{block}<4>{}%
    Separate \texttt{ipot} values for the two O atoms allows
    \textsc{feff} to compute different muffin tin radii -- a good
    thing for the oxygenyl ligand.
  \end{block}
  \begin{textblock*}{0.33\linewidth}(12.0\TPHorizModule,10.3\TPVertModule)%
    \only<2>{\includegraphics[width=\linewidth]{images/circle_x.png}}
  \end{textblock*}
\end{frame}

\begin{frame}
  \frametitle{The paths list}
  \begin{columns}[T]
    \begin{column}{0.5\linewidth}
      \begin{center}
        \includegraphics[width=0.85\linewidth]{images/feffinp.png}
      \end{center}
    \end{column}
    \begin{column}{0.5\linewidth}
      \begin{center}
        \includegraphics[width=0.85\linewidth]{images/paths.png}
      \end{center}
    \end{column}
  \end{columns}
  \begin{block}{Degeneracy margin}
    Fuzzy degeneracy ``collapses'' nearly degenerate paths, giving you
    fewer individual paths to worry about in the fit.
  \end{block}
\end{frame}

\begin{frame}
  \frametitle{Fit results}
  \begin{columns}[T]
    \begin{column}{0.5\linewidth}
      2 paths, 6 parameters -- pretty simple!

      \bigskip

      I get a very similar fit as the reference from GCA.

      \bigskip

      \begin{tabular}{lrr}
        \small parameter & GCA & Here \\
        \hline
        $S_0^2$         &   1.00(10) & 0.93(10) \\
        $R_{ax}$        &  1.80(1) & 1.78(1) \\
        $\sigma^2_{ax}$ &  0.001(1) & 0.002(1) \\
        $R_{eq}$        &  2.44(2) & 2.42(1) \\
        $\sigma^2_{eq}$ &  0.007(2) & 0.009(1) \\
      \end{tabular}
    \end{column}
    \begin{column}{0.5\linewidth}
      \includegraphics[width=0.85\linewidth]{images/fit.png}
    \end{column}
  \end{columns}
  \begin{bottomnote}[0.7][19.0]
    S.D.\ Kelly, et al., \textit{X-ray absorption fine structure
      determination of pH-dependent U-bacterial cell wall
      interactions}, Geochimica et Cosmochimica Acta \textbf{66}:22
    (2002) pp\  3855-3871.
    \doiref{10.1016/S0016-7037(02)00947-X}[LightBlue4]
  \end{bottomnote}
\end{frame}

\begin{frame}
  \frametitle{So, is it data?}
  \begin{columns}
    \begin{column}{0.6\linewidth}
      \includegraphics[width=0.85\linewidth]{images/fit.png}
    \end{column}
    \begin{column}{0.4\linewidth}
      There is clearly data beyond 9\,\AA$^{-1}$.

      \medskip

      In fact, I'd say there's data beyond 14\,\AA$^{-1}$!

      \medskip

      More measuring time would likely help in this case.
    \end{column}
  \end{columns}
\end{frame}

\end{document}

%%% Local Variables:
%%% mode: latex
%%% TeX-master: t
%%% TeX-parse-self: t
%%% TeX-auto-save: t
%%% TeX-auto-untabify: t
%%% TeX-PDF-mode: t
%%% End:
