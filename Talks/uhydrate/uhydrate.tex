\documentclass[10pt, xcolor=x11names, compress]{beamer}
%\documentclass[10pt, xcolor=x11names, compress, handout]{beamer}
\usetheme{progressbar}
%\usecolortheme[named=Purple4]{structure}
\progressbaroptions{headline=sections,titlepage=normal,frametitle=normal}

\setbeamertemplate{navigation symbols}{}

\usepackage{iwona} 

\usepackage{alltt}
\usepackage{amsmath,amsfonts, amssymb, amscd}
\usepackage{hyperref}
\usepackage{setspace}
\usepackage{wasysym}
\usepackage{ulem}
\usepackage{xspace}

\usepackage{calc}
\usepackage[overlay,absolute]{textpos}
\TPGrid[5mm,5mm]{20}{20}



\renewcommand{\Re}{\operatorname{Re}}
\renewcommand{\Im}{\operatorname{Im}}
\newcommand{\debye}{\operatorname{debye}}

\newcommand{\chik}{$\chi(k)$}
\newcommand{\chir}{$|\tilde{\chi}(R)|$}


\newcommand{\file}[1]{{\color{Firebrick4}\texttt{`#1'}}}
\newcommand{\multiple}{{\color{Orange3}\textsl{multiple}}}

\definecolor{Programs}{rgb}{0.0,0.2,0.0}
\newcommand{\atoms}    {{\color{Programs}\textsc{atoms}}\xspace}
\newcommand{\feff}     {{\color{Programs}\textsc{feff}}\xspace}
\newcommand{\feffex}   {{\color{Programs}\textsc{feff85exafs}}\xspace}
\newcommand{\feffsix}  {{\color{Programs}\textsc{feff6}}\xspace}
\newcommand{\feffeight}{{\color{Programs}\textsc{feff8}}\xspace}
\newcommand{\feffnine} {{\color{Programs}\textsc{feff9}}\xspace}
\newcommand{\ifeffit}  {{\color{Programs}\textsc{ifeffit}}\xspace}
\newcommand{\athena}   {{\color{Programs}\textsc{athena}}\xspace}
\newcommand{\artemis}  {{\color{Programs}\textsc{artemis}}\xspace}

\renewenvironment<>{center}
{\begin{actionenv}#1\begin{originalcenter}}
{\end{originalcenter}\end{actionenv}}

\definecolor{guessp}   {rgb}{0.64,0.00,0.64}
\newcommand{\guessp}   {{\color{guessp}guess}}
\definecolor{defp}     {rgb}{0.00,0.55,0.00}
\newcommand{\defp}     {{\color{defp}def}}
\definecolor{setp}     {rgb}{0,0,0}
\newcommand{\setp}     {{\color{setp}set}}
\definecolor{lguessp}  {rgb}{0.24,0.11,0.56}
\newcommand{\lguessp}  {{\color{lguessp}lguess}}
\definecolor{skipp}    {rgb}{0.70,0.70,0.70}
\newcommand{\skipp}    {{\color{skipp}skip}}
\definecolor{restrainp}{rgb}{0.80,0.61,0.11}
\newcommand{\restrainp}{{\color{restrainp}restrain}}
\definecolor{afterp}   {rgb}{0.29,0.44,0.55}
\newcommand{\afterp}   {{\color{afterp}after}}
\definecolor{penaltyp} {rgb}{0.55,0.35,0.17}
\newcommand{\penaltyp} {{\color{penaltyp}penalty}}
\definecolor{mergep}   {rgb}{0.93,0.00,0.00}
\newcommand{\mergep}   {{\color{mergep}merge}}

\hyphenation{EXAFS}

%% inline enumeration, see
%% http://tex.stackexchange.com/questions/94478/beamer-inline-itemize-and-enumeration/94521#94521
\newcounter{newenumi}
\setcounter{newenumi}{1}
\newcommand{\inlineenum}{%
 {%
 \setcounter{enumi}{\thenewenumi}%
 \leavevmode\usebeamertemplate{enumerate  item}
 \stepcounter{newenumi}
 \setcounter{enumi}{0}
 }
}
\newcommand{\resetinlineenum}{\setcounter{newenumi}{1}}

\usepackage{xparse}
\definecolor{bngray}{rgb}{0.5,0.5,0.5}
\NewDocumentEnvironment{bottomnote}{O{0.5}O{19.5}}%[0.85][19.5]
{\begin{textblock*}{#1\linewidth}(0pt,#2\TPVertModule)%
   \tiny\begin{color}{bngray}}%
{\end{color}\end{textblock*}}

\NewDocumentCommand{\cornerlogo}{m}
{\begin{textblock*}{0.08\linewidth}(17.2\TPHorizModule,0\TPVertModule)%
    \includegraphics[width=2cm]{#1}%
  \end{textblock*}} 


\DeclareDocumentCommand{\doiref}{m}%
{\href{http://dx.doi.org/#1}{\color{Blue4}DOI:~#1}}


%% define new commands here
%\newcommand{\eto}{EuTiO$_3$}

\mode<presentation>

\title{Noisy Data on Uranyl Hydrate}
%\subtitle{}

\author{Bruce Ravel}
\institute[NIST]{Synchrotron Science Group, Materials Measurement Science Division\\%
  Materials Measurement Laboratory\\%
  National Institute of Standards and Technology\\%
  \&\\%
  Beamline for Materials Measurements\\%
  National Synchrotron Light Source II\\~}


\date[Diamond2011]{EXAFS Data Analysis workshop 2011\\
  Diamond Light Source\\November 14--17, 2011\\~}

%\date[Diamond2011]{EXAFS Data Analysis workshop 2011\\
%  Diamond Light Source\\November 14--17, 2011}

\begin{document}
\maketitle

\begin{frame}
  \frametitle{Copyright}
  \tiny

  This document is copyright \copyright\ 2010-2015 Bruce Ravel.

  \begin{center}
    \includegraphics[width=1.0cm]{cc-by-sa.png}
  \end{center}

  This work is licensed under the Creative Commons
  Attribution-ShareAlike License.  To view a copy of this license,
  visit \href{http://creativecommons.org/licenses/by-sa/3.0/}
  {\color{Purple4}\texttt{http://creativecommons.org/licenses/by-sa/3.0/}}
  or send a letter to Creative Commons, 559 Nathan Abbott Way,
  Stanford, California 94305, USA.

  \begin{description}[Under the following conditions:]
  \tiny
  \item[You are free:] %
    \begin{itemize}
      \tiny
    \item \textbf{to Share} --- to copy, distribute, and transmit the work
    \item \textbf{to Remix} --- to adapt the work
    \item to make commercial use of the work
    \end{itemize}
  \item[Under the following conditions:] %
    \begin{itemize}
      \tiny
    \item \textbf{Attribution} -- You must attribute the work in the manner
      specified by the author or licensor (but not in any way that
      suggests that they endorse you or your use of the work).
    \item \textbf{Share Alike} -- If you alter, transform, or build upon this
      work, you may distribute the resulting work only under the same,
      similar or a compatible license.
    \end{itemize}
  \item[With the understanidng that:] 
    \begin{itemize}
      \tiny
    \item \textbf{Waiver} -- Any of the above conditions can be waived
      if you get permission from the copyright holder.
    \item \textbf{Public Domain} -- Where the work or any of its
      elements is in the public domain under applicable law, that
      status is in no way affected by the license.
    \item \textbf{Other Rights} -- In no way are any of the following
      rights affected by the license:
      \begin{itemize}
      \tiny
      \item Your fair dealing or fair use rights, or other
        applicable copyright exceptions and limitations;
      \item The author's moral rights;
      \item Rights other persons may have either in the work itself
        or in how the work is used, such as publicity or privacy
        rights.
      \end{itemize}
    \item \textbf{Notice} -- For any reuse or distribution, you must
      make clear to others the license terms of this work.
    \end{itemize}
  \end{description}

  This is a human-readable summary of the Legal Code (the full
  license).


\end{frame}

%%% Local Variables:
%%% mode: latex
%%% End:


\begin{frame}
  \frametitle{The sample}
  \begin{columns}[T]
    \begin{column}{0.5\linewidth}
      1000 ppm uranyl nitrate solution dissolved in DI water, pH=0.96\\[3ex]

      \includegraphics[width=\linewidth]{uranyl.pdf}

      \bigskip

      The liquid was held in a fluid cell and measured in fluorescence.
    \end{column}
    \begin{column}{0.5\linewidth}
      \includegraphics[width=\linewidth]{images/uhydrate_mu.png}

      \includegraphics[width=\linewidth]{images/uhydrate_chi.png}
    \end{column}
  \end{columns}

  \begin{textblock*}{0.7\linewidth}(0pt,19.0\TPVertModule)%
    \tiny%
    S.D.\ Kelly, et al., \textit{X-ray absorption fine structure
      determination of pH-dependent U-bacterial cell wall
      interactions}, Geochimica et Cosmochimica Acta \textbf{66}:22
    (2002) pp\  3855-3871.
    \href{http://dx.doi.org/10.1016/S0016-7037(02)00947-X}
    {\color{Blue4}\texttt{DOI:10.1016/S0016-7037(02)00947-X}}
  \end{textblock*}
\end{frame}

\begin{frame}
  \frametitle{Statistical noise and systematic noise}
  \begin{columns}[T]
    \begin{column}{0.5\linewidth}
      \includegraphics[width=\linewidth]{images/merged_chi.png}
      
    \end{column}
    \begin{column}{0.5\linewidth}
      \begin{tabular}{lr}
        data & \multicolumn{1}{c}{$\epsilon_k$} \\
        \hline
        5 scans & $\sim4.76\times10^{-3}$ \\
        merge   & $1.77\times10^{-3}$
      \end{tabular}
      $$\frac{\epsilon_1}{\epsilon_{merge}} = 2.69$$
      $$\sqrt{5} \approx 2.24$$
    \end{column}
  \end{columns}
  \begin{block}{The data seem show the behavior of statistcal noise}
    But is the signal at 14\,\AA$^{-1}$ really data?  In fact, is the
    signal beyond 9\,\AA$^{-1}$ really data?
  \end{block}
  \begin{alertblock}{}
    \centering In any case, how to we proceed with a molecule in solution?
  \end{alertblock}
\end{frame}

\begin{frame}[fragile]
  \frametitle{Sodium uranyl triacetate}
  \begin{columns}[T]
    \begin{column}{0.3\linewidth}
      \includegraphics[width=\linewidth]{../ATEA/mfc/NaU_triacetate_full.png}

      \includegraphics[width=\linewidth]{../ATEA/mfc/NaU_triacetate.png}
    \end{column}
    \begin{column}{0.7\linewidth}
      \begin{center}
        \begin{minipage}{0.75\linewidth}
          \begin{block}{}
            \tiny
            \begin{alltt}
{\color{Green4}title Na uranyl triacetate}
{\color{Green4}title Templeton, et al, Acta Cryst 1985 C41 1439-1441}
{\color{SteelBlue4}space} = P 21 3
{\color{SteelBlue4}rmax}  = 7.0    {\color{SteelBlue4}a}=10.689
{\color{SteelBlue4}core}  = U
{\color{Purple4}atoms}
{\color{Blue4}! At.type   x        y       z       tag}
  U     0.4294    0.4294    0.4294  U 
  Na    0.8286    0.8286    0.8286  Na
  O     0.3343    0.3343    0.3343  O1
  O     0.5242    0.5242    0.5242  O2
  O     0.3834    0.2945    0.6110  O3
  O     0.5464    0.2443    0.5007  O4
  C     0.4786    0.2260    0.5950  C1
  C     0.5088    0.1240    0.6862  C2 
            \end{alltt}
          \end{block}
        \end{minipage}
      \end{center}
    \end{column}
  \end{columns}
\end{frame}



\end{document}

%%% Local Variables:
%%% mode: latex
%%% TeX-master: t
%%% TeX-parse-self: t
%%% TeX-auto-save: t
%%% TeX-auto-untabify: t
%%% TeX-PDF-mode: t
%%% End:
