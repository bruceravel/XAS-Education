% see https://tex.stackexchange.com/a/74828
\PassOptionsToPackage{x11names}{xcolor}
\documentclass[10pt, xcolor=x11names, compress]{beamer}
%\documentclass[10pt, xcolor=x11names, compress, handout]{beamer}
\usetheme{progressbar}
%\usecolortheme[named=Purple4]{structure}
\progressbaroptions{headline=sections,titlepage=normal,frametitle=normal}

\setbeamertemplate{navigation symbols}{}

\usepackage{iwona} 

\usepackage{alltt}
\usepackage{amsmath,amsfonts, amssymb, amscd}
\usepackage{hyperref}
\usepackage{setspace}
\usepackage{wasysym}
\usepackage{ulem}
\usepackage{xspace}

\usepackage{calc}
\usepackage[overlay,absolute]{textpos}
\TPGrid[5mm,5mm]{20}{20}


\usepackage[x11names]{xcolor}

\usepackage{marvosym}
\newcommand{\homepagesymbol}{{\Large\ComputerMouse~}}%      {{\Large\marvosymbol{205}}~}


\renewcommand{\Re}{\operatorname{Re}}
\renewcommand{\Im}{\operatorname{Im}}
\newcommand{\debye}{\operatorname{debye}}

\newcommand{\chik}{$\chi(k)$}
\newcommand{\chir}{$|\tilde{\chi}(R)|$}


\newcommand{\file}[1]{{\color{Firebrick4}\texttt{`#1'}}}
\newcommand{\multiple}{{\color{Orange3}\textsl{multiple}}}

\definecolor{Programs}{rgb}{0.0,0.2,0.0}
\newcommand{\atoms}    {{\color{Programs}\textsc{atoms}}\xspace}
\newcommand{\feff}     {{\color{Programs}\textsc{feff}}\xspace}
\newcommand{\feffex}   {{\color{Programs}\textsc{feff85exafs}}\xspace}
\newcommand{\feffsix}  {{\color{Programs}\textsc{feff6}}\xspace}
\newcommand{\feffeight}{{\color{Programs}\textsc{feff8}}\xspace}
\newcommand{\feffnine} {{\color{Programs}\textsc{feff9}}\xspace}
\newcommand{\ifeffit}  {{\color{Programs}\textsc{ifeffit}}\xspace}
\newcommand{\larch}    {{\color{Programs}\textsc{larch}}\xspace}
\newcommand{\athena}   {{\color{Programs}\textsc{athena}}\xspace}
\newcommand{\artemis}  {{\color{Programs}\textsc{artemis}}\xspace}

\renewenvironment<>{center}
{\begin{actionenv}#1\begin{originalcenter}}
{\end{originalcenter}\end{actionenv}}

\definecolor{guessp}   {rgb}{0.64,0.00,0.64}
\newcommand{\guessp}   {{\color{guessp}guess}}
\definecolor{defp}     {rgb}{0.00,0.55,0.00}
\newcommand{\defp}     {{\color{defp}def}}
\definecolor{setp}     {rgb}{0,0,0}
\newcommand{\setp}     {{\color{setp}set}}
\definecolor{lguessp}  {rgb}{0.24,0.11,0.56}
\newcommand{\lguessp}  {{\color{lguessp}lguess}}
\definecolor{skipp}    {rgb}{0.70,0.70,0.70}
\newcommand{\skipp}    {{\color{skipp}skip}}
\definecolor{restrainp}{rgb}{0.80,0.61,0.11}
\newcommand{\restrainp}{{\color{restrainp}restrain}}
\definecolor{afterp}   {rgb}{0.29,0.44,0.55}
\newcommand{\afterp}   {{\color{afterp}after}}
\definecolor{penaltyp} {rgb}{0.55,0.35,0.17}
\newcommand{\penaltyp} {{\color{penaltyp}penalty}}
\definecolor{mergep}   {rgb}{0.93,0.00,0.00}
\newcommand{\mergep}   {{\color{mergep}merge}}

\hyphenation{EXAFS}

\newcommand{\exafsequation}[1][\small]{%
  {#1
    \begin{align*}
      \chi(k,\Gamma) =& 
      { \frac{{\color{Red4}(N_\Gamma S_0^2)}{\color{Blue4}F_\Gamma(k)}
                        e^{-2{\color{Red4}\sigma_\Gamma^2}k^2}
                        e^{-2R_\Gamma/{\color{Blue4}\lambda(k)}}
                        }
                        {2\,kR_\Gamma^2} }
      \sin{(2kR_\Gamma + {\color{Blue4}\Phi_\Gamma(k)})} \\
      \chi_{\mathrm{theory}}(k) =& \sum\limits_{\Gamma}\chi(k,\Gamma)\\
      R_\Gamma =& \> {\color{Blue4}R_{0,\Gamma}} +
      {\color{Red4}\Delta R_\Gamma} \\
      k =& \sqrt{2m_e(E_0 - {\color{Red4}\Delta E_0})/\hbar^2} 
           \approx \sqrt{(E_0 - {\color{Red4}\Delta E_0})/3.81}
    \end{align*}}
}


%% inline enumeration, see
%% http://tex.stackexchange.com/questions/94478/beamer-inline-itemize-and-enumeration/94521#94521
\newcounter{newenumi}
\setcounter{newenumi}{1}
\newcommand{\inlineenum}{%
 {%
 \setcounter{enumi}{\thenewenumi}%
 \leavevmode\usebeamertemplate{enumerate  item}
 \stepcounter{newenumi}
 \setcounter{enumi}{0}
 }
}
\newcommand{\resetinlineenum}{\setcounter{newenumi}{1}}

\usepackage{xparse}
\definecolor{bngray}{rgb}{0.5,0.5,0.5}
\NewDocumentEnvironment{bottomnote}{O{0.5}O{19.5}}%[0.85][19.5]
{\begin{textblock*}{#1\linewidth}(0pt,#2\TPVertModule)%
   \tiny\begin{color}{bngray}}%
{\end{color}\end{textblock*}}

\NewDocumentCommand{\cornerlogo}{m}
{\begin{textblock*}{0.08\linewidth}(17.2\TPHorizModule,0\TPVertModule)%
    \includegraphics[width=2cm]{#1}%
  \end{textblock*}} 
\NewDocumentCommand{\smcornerlogo}{m}
{\begin{textblock*}{0.08\linewidth}(19.5\TPHorizModule,0\TPVertModule)%
    \includegraphics[width=7mm]{#1}%
  \end{textblock*}} 


\DeclareDocumentCommand{\doiref}{mO{Blue4}}%
{\href{https://doi.org/#1}{\color{#2}{\ComputerMouse~}DOI:~#1}}

\DeclareDocumentCommand{\inlinelogo}{m}%
{\raisebox{-.2\height}{\includegraphics[width=5mm]{#1}}\xspace}

\DeclareDocumentCommand{\titlepageurl}{mO{Blue4}}%
{~\\[2ex]{\footnotesize\href{#1}{\color{#2}%
    {\ComputerMouse\,}PDF of this talk: #1}}}

%% define new commands here
%\newcommand{\eto}{EuTiO$_3$}

\mode<presentation>

\title{Mechanics of an XAS Scan}
%\subtitle{}
\include{author}
%\include{date}
\date[\today]{\today}

\begin{document}
\maketitle

\begin{frame}
  \frametitle{Copyright}
  \tiny

  This document is copyright \copyright\ 2010-2015 Bruce Ravel.

  \begin{center}
    \includegraphics[width=1.0cm]{cc-by-sa.png}
  \end{center}

  This work is licensed under the Creative Commons
  Attribution-ShareAlike License.  To view a copy of this license,
  visit \href{http://creativecommons.org/licenses/by-sa/3.0/}
  {\color{Purple2}\texttt{http://creativecommons.org/licenses/by-sa/3.0/}}
  or send a letter to Creative Commons, 559 Nathan Abbott Way,
  Stanford, California 94305, USA.

  \begin{description}[Under the following conditions:]
  \tiny
  \item[You are free:] %
    \begin{itemize}
      \tiny
    \item \textbf{to Share} --- to copy, distribute, and transmit the work
    \item \textbf{to Remix} --- to adapt the work
    \item to make commercial use of the work
    \end{itemize}
  \item[Under the following conditions:] %
    \begin{itemize}
      \tiny
    \item \textbf{Attribution} -- You must attribute the work in the manner
      specified by the author or licensor (but not in any way that
      suggests that they endorse you or your use of the work).
    \item \textbf{Share Alike} -- If you alter, transform, or build upon this
      work, you may distribute the resulting work only under the same,
      similar or a compatible license.
    \end{itemize}
  \item[With the understanidng that:] 
    \begin{itemize}
      \tiny
    \item \textbf{Waiver} -- Any of the above conditions can be waived
      if you get permission from the copyright holder.
    \item \textbf{Public Domain} -- Where the work or any of its
      elements is in the public domain under applicable law, that
      status is in no way affected by the license.
    \item \textbf{Other Rights} -- In no way are any of the following
      rights affected by the license:
      \begin{itemize}
      \tiny
      \item Your fair dealing or fair use rights, or other
        applicable copyright exceptions and limitations;
      \item The author's moral rights;
      \item Rights other persons may have either in the work itself
        or in how the work is used, such as publicity or privacy
        rights.
      \end{itemize}
    \item \textbf{Notice} -- For any reuse or distribution, you must
      make clear to others the license terms of this work.
    \end{itemize}
  \end{description}

  This is a human-readable summary of the Legal Code (the full
  license).


\end{frame}

%%% Local Variables:
%%% mode: latex
%%% End:


\begin{frame}
  \frametitle{Good data starts with a good plan}
  
\end{frame}

\begin{frame}
  \frametitle{Two scan types}
  \begin{description}
  \item[Step scan] blah blah
  \item[Slew scan] Blah blah
  \end{description}
\end{frame}

\section{Step scanning}

\begin{frame}
  \frametitle{XANES and EXAFS}
  
\end{frame}

\begin{frame}
  \frametitle{Scan parameters}
  \begin{enumerate}
  \item Regions
  \item Step sizes
  \item Integration times
  \end{enumerate}
\end{frame}

\begin{frame}
  \frametitle{Considerations for data processing}
  Show a figure of canonical XAS data (some Fe oxide) with pre- and
  post edge lines.  Overplot with a severely truncated version of the
  same.  Discuss the difficulties of normalization and its impact on
  subsequent interpretation.
\end{frame}

\section{Slew scanning}

\begin{frame}
  \frametitle{Slew...}
  
\end{frame}





\end{document}

%%% Local Variables:
%%% mode: latex
%%% TeX-master: t
%%% TeX-parse-self: t
%%% TeX-auto-save: t
%%% TeX-auto-untabify: t
%%% TeX-PDF-mode: t
%%% End:
\grid
