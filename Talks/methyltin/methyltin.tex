%\documentclass[10pt, xcolor=x11names, compress]{beamer}
\documentclass[10pt, xcolor=x11names, compress, handout]{beamer}

\usetheme{progressbar}
%\usecolortheme[named=Purple4]{structure}
\progressbaroptions{headline=sections,titlepage=normal,frametitle=normal}

\setbeamertemplate{navigation symbols}{}

\usepackage{iwona} 

\usepackage{alltt}
\usepackage{amsmath,amsfonts, amssymb, amscd}
\usepackage{hyperref}
\usepackage{setspace}
\usepackage{wasysym}
\usepackage{ulem}
\usepackage{xspace}

\usepackage{calc}
\usepackage[overlay,absolute]{textpos}
\TPGrid[5mm,5mm]{20}{20}



\renewcommand{\Re}{\operatorname{Re}}
\renewcommand{\Im}{\operatorname{Im}}
\newcommand{\debye}{\operatorname{debye}}

\newcommand{\chik}{$\chi(k)$}
\newcommand{\chir}{$|\tilde{\chi}(R)|$}


\newcommand{\file}[1]{{\color{Firebrick4}\texttt{`#1'}}}
\newcommand{\multiple}{{\color{Orange3}\textsl{multiple}}}

\definecolor{Programs}{rgb}{0.0,0.2,0.0}
\newcommand{\atoms}    {{\color{Programs}\textsc{atoms}}\xspace}
\newcommand{\feff}     {{\color{Programs}\textsc{feff}}\xspace}
\newcommand{\feffex}   {{\color{Programs}\textsc{feff85exafs}}\xspace}
\newcommand{\feffsix}  {{\color{Programs}\textsc{feff6}}\xspace}
\newcommand{\feffeight}{{\color{Programs}\textsc{feff8}}\xspace}
\newcommand{\feffnine} {{\color{Programs}\textsc{feff9}}\xspace}
\newcommand{\ifeffit}  {{\color{Programs}\textsc{ifeffit}}\xspace}
\newcommand{\athena}   {{\color{Programs}\textsc{athena}}\xspace}
\newcommand{\artemis}  {{\color{Programs}\textsc{artemis}}\xspace}

\renewenvironment<>{center}
{\begin{actionenv}#1\begin{originalcenter}}
{\end{originalcenter}\end{actionenv}}

\definecolor{guessp}   {rgb}{0.64,0.00,0.64}
\newcommand{\guessp}   {{\color{guessp}guess}}
\definecolor{defp}     {rgb}{0.00,0.55,0.00}
\newcommand{\defp}     {{\color{defp}def}}
\definecolor{setp}     {rgb}{0,0,0}
\newcommand{\setp}     {{\color{setp}set}}
\definecolor{lguessp}  {rgb}{0.24,0.11,0.56}
\newcommand{\lguessp}  {{\color{lguessp}lguess}}
\definecolor{skipp}    {rgb}{0.70,0.70,0.70}
\newcommand{\skipp}    {{\color{skipp}skip}}
\definecolor{restrainp}{rgb}{0.80,0.61,0.11}
\newcommand{\restrainp}{{\color{restrainp}restrain}}
\definecolor{afterp}   {rgb}{0.29,0.44,0.55}
\newcommand{\afterp}   {{\color{afterp}after}}
\definecolor{penaltyp} {rgb}{0.55,0.35,0.17}
\newcommand{\penaltyp} {{\color{penaltyp}penalty}}
\definecolor{mergep}   {rgb}{0.93,0.00,0.00}
\newcommand{\mergep}   {{\color{mergep}merge}}

\hyphenation{EXAFS}

%% inline enumeration, see
%% http://tex.stackexchange.com/questions/94478/beamer-inline-itemize-and-enumeration/94521#94521
\newcounter{newenumi}
\setcounter{newenumi}{1}
\newcommand{\inlineenum}{%
 {%
 \setcounter{enumi}{\thenewenumi}%
 \leavevmode\usebeamertemplate{enumerate  item}
 \stepcounter{newenumi}
 \setcounter{enumi}{0}
 }
}
\newcommand{\resetinlineenum}{\setcounter{newenumi}{1}}

\usepackage{xparse}
\definecolor{bngray}{rgb}{0.5,0.5,0.5}
\NewDocumentEnvironment{bottomnote}{O{0.5}O{19.5}}%[0.85][19.5]
{\begin{textblock*}{#1\linewidth}(0pt,#2\TPVertModule)%
   \tiny\begin{color}{bngray}}%
{\end{color}\end{textblock*}}

\NewDocumentCommand{\cornerlogo}{m}
{\begin{textblock*}{0.08\linewidth}(17.2\TPHorizModule,0\TPVertModule)%
    \includegraphics[width=2cm]{#1}%
  \end{textblock*}} 


\DeclareDocumentCommand{\doiref}{m}%
{\href{http://dx.doi.org/#1}{\color{Blue4}DOI:~#1}}


\mode<presentation>

\title{Methyltin EXAFS}%
\subtitle{An Artemis example}

\author{Bruce Ravel}
\institute[NIST]{Synchrotron Science Group, Materials Measurement Science Division\\%
  Materials Measurement Laboratory\\%
  National Institute of Standards and Technology\\%
  \&\\%
  Beamline for Materials Measurements\\%
  National Synchrotron Light Source II\\~}


\date{\today}

\begin{document}

\maketitle

\begin{frame}
  \frametitle{Copyright}
  \tiny

  This document is copyright \copyright\ 2010-2015 Bruce Ravel.

  \begin{center}
    \includegraphics[width=1.0cm]{cc-by-sa.png}
  \end{center}

  This work is licensed under the Creative Commons
  Attribution-ShareAlike License.  To view a copy of this license,
  visit \href{http://creativecommons.org/licenses/by-sa/3.0/}
  {\color{Purple4}\texttt{http://creativecommons.org/licenses/by-sa/3.0/}}
  or send a letter to Creative Commons, 559 Nathan Abbott Way,
  Stanford, California 94305, USA.

  \begin{description}[Under the following conditions:]
  \tiny
  \item[You are free:] %
    \begin{itemize}
      \tiny
    \item \textbf{to Share} --- to copy, distribute, and transmit the work
    \item \textbf{to Remix} --- to adapt the work
    \item to make commercial use of the work
    \end{itemize}
  \item[Under the following conditions:] %
    \begin{itemize}
      \tiny
    \item \textbf{Attribution} -- You must attribute the work in the manner
      specified by the author or licensor (but not in any way that
      suggests that they endorse you or your use of the work).
    \item \textbf{Share Alike} -- If you alter, transform, or build upon this
      work, you may distribute the resulting work only under the same,
      similar or a compatible license.
    \end{itemize}
  \item[With the understanidng that:] 
    \begin{itemize}
      \tiny
    \item \textbf{Waiver} -- Any of the above conditions can be waived
      if you get permission from the copyright holder.
    \item \textbf{Public Domain} -- Where the work or any of its
      elements is in the public domain under applicable law, that
      status is in no way affected by the license.
    \item \textbf{Other Rights} -- In no way are any of the following
      rights affected by the license:
      \begin{itemize}
      \tiny
      \item Your fair dealing or fair use rights, or other
        applicable copyright exceptions and limitations;
      \item The author's moral rights;
      \item Rights other persons may have either in the work itself
        or in how the work is used, such as publicity or privacy
        rights.
      \end{itemize}
    \item \textbf{Notice} -- For any reuse or distribution, you must
      make clear to others the license terms of this work.
    \end{itemize}
  \end{description}

  This is a human-readable summary of the Legal Code (the full
  license).


\end{frame}

%%% Local Variables:
%%% mode: latex
%%% End:


\section{Background}

\begin{frame}
  \frametitle{The science}
  \footnotesize%
  Polyvinyl chloride -- PVC -- is a type of rigid plastic used for
  water and sewage transport in the United States and elsewhere.  In
  the US, home building codes require copper pipe for bringing water
  into homes and PVC for carrying water and sewage out of home.

  \medskip

  The manufacture of PVC uses organic tin species, mostly dimethyl
  tin, as a stabilizing agent.  Over time, organic tin species can
  leach out of PVC and into municipal water supplies.  My collaborator,
  Chris Impellitteri, at the US Environmental Protection Agency studied
  this leaching process.  The organic tin species used as stabilizers
  evolve during the manufacturing process, so we used XAS to identify
  and characterize the tin species present in commercial PVC pipes.

  \medskip

  To start, we have built a library of organic (methyl tin, butyl tin,
  phenyl tin, tricyclohexyl tin, etc) and inorganic (metallic tin, tin
  oxide, tin chloride) standard compounds.  To have confidence that we
  could interpret spectra measured on PVC samples, we first carefully
  analyze the standards.  In this document, I show the analysis of two
  methyl tin species.

  \medskip

  As well as being a real-world example of using Artemis, this will
  serve to introduce several important concepts, including running
  {\feff} from a molecule rather than a crystal, multiple data set
  fitting, and the concept of constraining parameters across data
  sets.

  \medskip

  This document is intended as a supplement to the
  demonstration/lecture on the same topic that I give as a part of an
  XAS training course.

  \medskip

  ~

  \begin{textblock*}{0.7\linewidth}(0pt,18.75\TPVertModule)%
    \tiny%
    C.\ Impellitteri, et al., \textit{Speciation of organotins in
      polyvinyl chloride pipe via X-ray absorption spectroscopy and in
      leachates using GC-PFPD after derivatisation}, Journal of
    Environmental Monitoring \textbf{9} (2007) pp\ 358-365.
    \href{http://dx.doi.org/10.1039/B617711E}
    {\color{Blue4}\texttt{DOI:10.1039/B617711E}}
  \end{textblock*}
\end{frame}

\begin{frame}
  \frametitle{Methyl tin chloride}
  The samples we will examine are two methyl tin chloride species
  dissolved in an organic solvent.

  \begin{columns}
    \begin{column}{0.5\linewidth}
      \begin{center}
        \includegraphics[width=0.5\linewidth]{../noxtal/images/dimethyltin_dichloride.png}\\
      Dimethyl tin dichloride
      \end{center}
    \end{column}
    \begin{column}{0.5\linewidth}
      \begin{center}
        \includegraphics[width=0.5\linewidth]{../noxtal/images/monomethyltin_trichloride.png}\\
        Monomethyl tin trichloride
      \end{center}
    \end{column}
  \end{columns}
  \begin{columns}
    \begin{column}{0.33\linewidth}
      \includegraphics[width=\linewidth]{../noxtal/images/mtin_mu.png}
    \end{column}
    \begin{column}{0.33\linewidth}
      \includegraphics[width=\linewidth]{../noxtal/images/mtin_chik.png}
    \end{column}
    \begin{column}{0.33\linewidth}
      \includegraphics[width=\linewidth]{../noxtal/images/mtin_chir.png}
    \end{column}
  \end{columns}
\end{frame}


\begin{frame}[fragile]
  \frametitle{Protein Data Bank file format}
  A bit of googling turned up a structure for dimethyl tin dichloride
  in the form of a PDB file.  It looks like this:

  \begin{block}{}
\begin{alltt}
\scriptsize
COMPND    5261536
HETATM    1 \alert{ C}1  LIG     1      \alert{-0.027   2.146   0.014}  1.00  0.00
HETATM    2 \alert{SN}2  LIG     1      \alert{ 0.002  -0.004   0.002}  1.00  0.00
HETATM    3 \alert{ C}3  LIG     1      \alert{ 1.042  -0.716   1.744}  1.00  0.00
HETATM    4 \alert{CL}4  LIG     1      \alert{-2.212  -0.821   0.019}  1.00  0.00
HETATM    5 \alert{CL}5  LIG     1      \alert{ 1.107  -0.765  -1.940}  1.00  0.00
HETATM    6 1\alert{H}1  LIG     1      \alert{ 0.996   2.523   0.006}  1.00  0.00
HETATM    7 2\alert{H}1  LIG     1      \alert{-0.554   2.507  -0.869}  1.00  0.00
HETATM    8 3\alert{H}1  LIG     1      \alert{-0.537   2.497   0.911}  1.00  0.00
HETATM    9 1\alert{H}3  LIG     1      \alert{ 0.532  -0.365   2.641}  1.00  0.00
HETATM   10 2\alert{H}3  LIG     1      \alert{ 1.057  -1.806   1.738}  1.00  0.00
HETATM   11 3\alert{H}3  LIG     1      \alert{ 2.065  -0.339   1.736}  1.00  0.00
END
\end{alltt}
  \end{block}

The \alert{red bits} are atomic species and cartesian coordinates ---
just what we need!
\end{frame}

\begin{frame}[fragile]
  \frametitle{Feff6 input file}
  \begin{columns}
    \begin{column}{0.45\linewidth}
\begin{alltt}
\scriptsize
 {\color{Green4}TITLE dimethyltin dichloride}
 {\color{Purple2}HOLE}      1   1.0  {\color{Blue4} *  Sn K edge (29200 eV), S0^2}
 {\color{Blue4}*         mphase,mpath,mfeff,mchi}
 {\color{SteelBlue2}CONTROL}   1      1     1     1
 {\color{SteelBlue2}PRINT}     1      0     0     0
 {\color{Purple2}RMAX}      6.0

 {\color{Brown4}POTENTIALS}
 {\color{Blue4}*    ipot   Z  element}
        0   50   Sn        
        1   17   Cl
        2    6   C
        3    1   H
 {\color{Brown4}ATOMS}
 {\color{Blue4}*   x       y       z    ipot}
   -0.027   2.146   0.014  2
    0.002  -0.004   0.002  0
    1.042  -0.716   1.744  2
   -2.212  -0.821   0.019  1
    1.107  -0.765  -1.940  1
    0.996   2.523   0.006  3
   -0.554   2.507  -0.869  3
   -0.537   2.497   0.911  3
    0.532  -0.365   2.641  3
    1.057  -1.806   1.738  3
    2.065  -0.339   1.736  3
\end{alltt}      
    \end{column}
    \begin{column}{0.55\linewidth}
      ~\\[2ex]

      \begin{enumerate}
      \item Prepare \file{feff.inp} boilerplate
      \item Cut-n-paste the cartesian coordinates in the
        {\color{Brown4}\texttt{ATOMS}} list
      \item Make a {\color{Brown4}\texttt{POTENTIALS}} list out the
        atomic species
      \item The absorber \textbf{must} be potential \#0, but it need
        be neither first in the {\color{Brown4}\texttt{ATOMS}} list
        nor be at (0,0,0)
      \item The {\color{Brown4}\texttt{ATOMS}} list need not be in
        order of radial distance (or any other order)
      \item This \file{feff.inp} file can be imported directly into
        {\artemis}
      \end{enumerate}
    \end{column}
  \end{columns}
\end{frame}


\section{Simple fitting model}

\begin{frame}
  \frametitle{Create a simple fitting model}
  \footnotesize%
  \begin{enumerate}
  \item Import the dimethytin dichloride (DMT) data from the {\athena}
    project file
  \item Import the \file{feff.inp} file for DMT
  \item Run {\feff} then drag and drop the first two paths
    (Sn$\leftrightarrows$C and Sn$\leftrightarrows$Cl) onto the
    DMT data.
  \item Create guess parameters for an overall amplitude and an
    overall E$_0$ shift.
  \item We cannot expect to share $\sigma^2$ or $\Delta R$ 
    between the C and Cl scatterers, so create 4 more parameters for
    those.  That comes to \alert{6} {\guessp} parameters.
  \item With the k-range set to [2:10.5] and the R-range set to
    [1:2.4], we have at most about \alert{7.5} independent points.
  \end{enumerate}
  \begin{columns}
    \begin{column}{0.4\linewidth}
      \includegraphics[width=\linewidth]{images/dmt_start.png}      
    \end{column}
    \begin{column}{0.6\linewidth}
      Guess 1 for the amplitude, 0 for both $\Delta R$ parameters and
      0.003 for both $\sigma^2$ parameters.
    \end{column}
  \end{columns}
\end{frame}

\begin{frame}[fragile]
  \frametitle{Results of the first fit}
  \small
  The fit doesn't seem bad.  The {\color{Red3}red line} over-plots the
  {\color{Blue3}blue line} rather well.  Unfortunately, the amplitude
  and both $\sigma^2$ parameters are suspiciously large, and one
  correlation is quite alarming.
  \begin{columns}
    \begin{column}{0.5\linewidth}
      \includegraphics[width=\linewidth]{images/dmt_kw2_fit.png}
    \end{column}
    \begin{column}{0.5\linewidth}
\begin{alltt}
\tiny 
\alert{Independent points          : 7.426757813
Number of variables         : 6}
Chi-square                  : 10523.027205554
Reduced chi-square          : 7375.482449341
R-factor                    : 0.012603127   
         
guess parameters:
  \alert{amp           =   3.17612332    # +/-   1.20737984}
  enot          =   5.48632866    # +/-   3.83329304
  dr_c          =   0.15998621    # +/-   0.10183822
  dr_cl         =   0.00886040    # +/-   0.02941155
  \alert{ss_c          =   0.04405173    # +/-   0.02749584}
  \alert{ss_cl         =   0.01784104    # +/-   0.00499746}

Correlations between variables:
          \alert{ss_cl & amp            -->  0.9231}
          dr_cl & enot           -->  0.8694
           dr_c & amp            -->  0.7677
           ss_c & enot           --> -0.6554
          ss_cl & dr_c           -->  0.6873
           ss_c & amp            -->  0.6070   

\end{alltt}
    \end{column}
  \end{columns}
  These data are severely stressed by fitting 6 parameters with barely
  more information.  That is the likely cause of the odd results.
\end{frame}

\begin{frame}[fragile]
  \frametitle{An unstable fit}
  \small%
  There is an even worse aspect of the fit -- it turns out to be
  unstable.  The result we just found is some kind of local minimum,
  but perhaps not the best fit.  Guessing 0.02 for \texttt{dr\_cl}
  results in the following:
  \begin{columns}
    \begin{column}{0.5\linewidth}
      \includegraphics[width=\linewidth]{images/dmt_better.png}      
    \end{column}
    \begin{column}{0.5\linewidth}
\begin{alltt}
\tiny
\alert{Independent points          : 7.426757813
Number of variables         : 6}
Chi-square                  : 16890.423023572
Reduced chi-square          : 11838.325240341
R-factor                    : 0.016206958

guess parameters:
  {\color{Green4}amp           =   1.19951643    # +/-   0.27919514}
  enot          =   3.72054447    # +/-   2.58474466
  dr_c          =  -0.06264818    # +/-   0.04220613
  dr_cl         =   0.01464366    # +/-   0.02710054
  {\color{Green4}ss_c          =   0.00208373    # +/-   0.00627642
  ss_cl         =   0.00506975    # +/-   0.00429784}

Correlations between variables:
          dr_cl & enot          -->  0.8889
          ss_cl & ss_c          -->  0.8785
          ss_cl & amp           -->  0.8698
           dr_c & enot          -->  0.8547
           ss_c & amp           -->  0.8429
          dr_cl & dr_c          -->  0.8047

\end{alltt}
    \end{column}
  \end{columns}
  This is an improvement in that the amplitude and the $\sigma^2$
  values are much more in line with what we expect, but correlations
  remain quite high.

  \medskip

  The next trick to try is a \alert{multiple data set} fit.
\end{frame}

\section{Multiple data set fit}


\begin{frame}
  \frametitle{Setting up a multiple data set fit}
  Import the monomethyltin trichloride (MMT) from the {\athena}
  project file.  This will open a second Data window and place a
  second item in the list of data sets.
  \begin{center}
    \includegraphics[width=0.7\linewidth]{images/mds.png}
  \end{center}
\end{frame}

\begin{frame}
  \frametitle{Cloning paths from DMT to MMT}
  \small
  Drag and drop both paths from the DMT window to the MMT window.
  Path drag and drop works by clicking on a path in the path list of
  the source \textit{while holding down the control key}.

  \begin{center}
    \includegraphics[width=0.65\linewidth]{images/MMT.png}
  \end{center}
  
  Change the N of the Sn$\leftrightarrows$C path to 1, since
  monomethyltin only has one methyl ligand.  Similarly, change the N
  of the Sn$\leftrightarrows$Cl path to 3, since there are three Cl ligands.
\end{frame}

\newtheorem{assumption}[theorem]{Assumption}

\begin{frame}
  \frametitle{Discussion}
  \begin{assumption}
    The Sn--C and Sn--Cl bonds are identical in DMT and MMT, thus we
    can use the same $\sigma^2$ and $\Delta R$ parameters for each
    data set.
  \end{assumption}

  Given this assumption, the fitting situation is much improved.  We
  have \alert{doubled} the information content while introducing
  \alert{0} additional parameters!  Both data sets contribute to the
  determination of our {\guessp} parameters.

  \begin{columns}
    \begin{column}{0.5\linewidth}
      \includegraphics[width=\linewidth]{images/MDS_fit.png}
    \end{column}
    \begin{column}{0.5\linewidth}
      The best fit values are much the same as for the better single
      data set fit.  The fit, however, is more stable and independent
      of the starting values.  The correlations are mostly smaller.
    \end{column}
  \end{columns}
\end{frame}

\begin{frame}
  \frametitle{What's next?}
  \small
  \begin{enumerate}
  \item Could the Fourier transform range be longer?  Look at the k123
    plot for each data set.  (I changed the k-range before making the
    plot on the previous page.)
  \item Could the fitting range be longer?  Well, there is not much
    signal beyond the first shell above the noise level.  Simply
    expanding the R-range to make $N_{idp}$ larger without actually
    including paths in that contribute spectral weight in the extended
    range is cheating.
  \item Is the assumption about the bonds in the two samples valid?
    How would you go about testing that assumption?
  \item Trimethyltin monochloride would have been a useful
    measurement....
  \item The $\Delta R$s for both Sn$\leftrightarrows$C and the
    Sn$\leftrightarrows$Cl are somewhat large.  The fit might be
    improved by adjusting the original \file{feff.inp}, re-running
    {\feff}, and re-doing the fit.
  \item The structure used in the {\feff} calculation is unbounded
    from the outside, which might effect the construction of muffin
    tins.  Packing water moelcues around the DMT molecule might help.
  \item Is the DMT {\feff} calculation transferable to MMT?  Running
    {\feff} on MMT might help.
  \end{enumerate}
\end{frame}
\end{document}


%%% Local Variables:
%%% TeX-parse-self: t
%%% TeX-auto-save: t
%%% TeX-auto-untabify: t
%%% TeX-PDF-mode: t
%%% End:
