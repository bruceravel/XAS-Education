\documentclass[10pt, xcolor=x11names, compress]{beamer}
%\documentclass[10pt, xcolor=x11names, compress, handout]{beamer}
\usetheme{progressbar}
%\usecolortheme[named=Purple4]{structure}
\progressbaroptions{headline=sections,titlepage=normal,frametitle=normal}

\setbeamertemplate{navigation symbols}{}

\usepackage{iwona} 

\usepackage{alltt}
\usepackage{amsmath,amsfonts, amssymb, amscd}
\usepackage{hyperref}
\usepackage{setspace}
\usepackage{wasysym}
\usepackage{ulem}
\usepackage{xspace}

\usepackage{calc}
\usepackage[overlay,absolute]{textpos}
\TPGrid[5mm,5mm]{20}{20}



\renewcommand{\Re}{\operatorname{Re}}
\renewcommand{\Im}{\operatorname{Im}}
\newcommand{\debye}{\operatorname{debye}}

\newcommand{\chik}{$\chi(k)$}
\newcommand{\chir}{$|\tilde{\chi}(R)|$}


\newcommand{\file}[1]{{\color{Firebrick4}\texttt{`#1'}}}
\newcommand{\multiple}{{\color{Orange3}\textsl{multiple}}}

\definecolor{Programs}{rgb}{0.0,0.2,0.0}
\newcommand{\atoms}    {{\color{Programs}\textsc{atoms}}\xspace}
\newcommand{\feff}     {{\color{Programs}\textsc{feff}}\xspace}
\newcommand{\feffex}   {{\color{Programs}\textsc{feff85exafs}}\xspace}
\newcommand{\feffsix}  {{\color{Programs}\textsc{feff6}}\xspace}
\newcommand{\feffeight}{{\color{Programs}\textsc{feff8}}\xspace}
\newcommand{\feffnine} {{\color{Programs}\textsc{feff9}}\xspace}
\newcommand{\ifeffit}  {{\color{Programs}\textsc{ifeffit}}\xspace}
\newcommand{\athena}   {{\color{Programs}\textsc{athena}}\xspace}
\newcommand{\artemis}  {{\color{Programs}\textsc{artemis}}\xspace}

\renewenvironment<>{center}
{\begin{actionenv}#1\begin{originalcenter}}
{\end{originalcenter}\end{actionenv}}

\definecolor{guessp}   {rgb}{0.64,0.00,0.64}
\newcommand{\guessp}   {{\color{guessp}guess}}
\definecolor{defp}     {rgb}{0.00,0.55,0.00}
\newcommand{\defp}     {{\color{defp}def}}
\definecolor{setp}     {rgb}{0,0,0}
\newcommand{\setp}     {{\color{setp}set}}
\definecolor{lguessp}  {rgb}{0.24,0.11,0.56}
\newcommand{\lguessp}  {{\color{lguessp}lguess}}
\definecolor{skipp}    {rgb}{0.70,0.70,0.70}
\newcommand{\skipp}    {{\color{skipp}skip}}
\definecolor{restrainp}{rgb}{0.80,0.61,0.11}
\newcommand{\restrainp}{{\color{restrainp}restrain}}
\definecolor{afterp}   {rgb}{0.29,0.44,0.55}
\newcommand{\afterp}   {{\color{afterp}after}}
\definecolor{penaltyp} {rgb}{0.55,0.35,0.17}
\newcommand{\penaltyp} {{\color{penaltyp}penalty}}
\definecolor{mergep}   {rgb}{0.93,0.00,0.00}
\newcommand{\mergep}   {{\color{mergep}merge}}

\hyphenation{EXAFS}

%% inline enumeration, see
%% http://tex.stackexchange.com/questions/94478/beamer-inline-itemize-and-enumeration/94521#94521
\newcounter{newenumi}
\setcounter{newenumi}{1}
\newcommand{\inlineenum}{%
 {%
 \setcounter{enumi}{\thenewenumi}%
 \leavevmode\usebeamertemplate{enumerate  item}
 \stepcounter{newenumi}
 \setcounter{enumi}{0}
 }
}
\newcommand{\resetinlineenum}{\setcounter{newenumi}{1}}

\usepackage{xparse}
\definecolor{bngray}{rgb}{0.5,0.5,0.5}
\NewDocumentEnvironment{bottomnote}{O{0.5}O{19.5}}%[0.85][19.5]
{\begin{textblock*}{#1\linewidth}(0pt,#2\TPVertModule)%
   \tiny\begin{color}{bngray}}%
{\end{color}\end{textblock*}}

\NewDocumentCommand{\cornerlogo}{m}
{\begin{textblock*}{0.08\linewidth}(17.2\TPHorizModule,0\TPVertModule)%
    \includegraphics[width=2cm]{#1}%
  \end{textblock*}} 


\DeclareDocumentCommand{\doiref}{m}%
{\href{http://dx.doi.org/#1}{\color{Blue4}DOI:~#1}}

\usepackage{chemfig}
\usepackage{mol2chemfig}

%% define new commands here
%\newcommand{\eto}{EuTiO$_3$}

%% this makes the dashed lines to the Hg atoms
\newcommand*{\bondwidth}{0.17 em} %'Bond Width'
\newcommand*{\bondboldwidth}{0.22832 em} %'Bold Width'
\newcommand*{\bondhashlength}{0.25737 em} % 'Hash Spacing'
\tikzset{
  bold bond/.style = {line width = \bondboldwidth},
  dash bond/.style =
    {dash pattern = on \bondhashlength off \bondhashlength},
  hash bond/.style =
    {
      dash pattern = on \bondwidth off \bondhashlength,
      line width   = \bondboldwidth
    },
}

\newcommand{\qm}{?}
\newcommand{\spc}{~}

\newcommand\namebond[4][5pt]{\chemmove{\path(#2)--(#3)node[midway,sloped,yshift=#1]{#4};}}
\newcommand\arcbetweennodes[3]{%
  \pgfmathanglebetweenpoints{\pgfpointanchor{#1}{center}}{\pgfpointanchor{#2}{center}}%
  \let#3\pgfmathresult}
\newcommand\arclabel[6][stealth-stealth,shorten <=1pt,shorten >=1pt]{%
  \chemmove{%
    \arcbetweennodes{#4}{#3}\anglestart \arcbetweennodes{#4}{#5}\angleend
    \draw[#1]([shift=(\anglestart:#2)]#4)arc(\anglestart:\angleend:#2);
    \pgfmathparse{(\anglestart+\angleend)/2}\let\anglestart\pgfmathresult
    \node[shift=(\anglestart:#2+4pt)#4,anchor=\anglestart+180,rotate=\anglestart+90,inner sep=0pt,
    outer sep=0pt]at(#4){#6};}}




\mode<presentation>

\title{A challenging EXAFS analysis problem}
%\subtitle{}

\author{Bruce Ravel}
\institute[NIST]{Synchrotron Science Group, Materials Measurement Science Division\\%
  Materials Measurement Laboratory\\%
  National Institute of Standards and Technology\\%
  \&\\%
  Beamline for Materials Measurements\\%
  National Synchrotron Light Source II\\~}


%\date[Diamond2011]{EXAFS Data Analysis workshop 2011\\
  Diamond Light Source\\November 14--17, 2011\\~}

\date[ACXAS 2014]{ASEAN Workshop on X-ray Absorption Spectroscopy\\
  Synchrotron Light Research Institute\\June 2--4, 2014}

\begin{document}
\maketitle

\begin{frame}
  \frametitle{Copyright}
  \tiny

  This document is copyright \copyright\ 2010-2015 Bruce Ravel.

  \begin{center}
    \includegraphics[width=1.0cm]{cc-by-sa.png}
  \end{center}

  This work is licensed under the Creative Commons
  Attribution-ShareAlike License.  To view a copy of this license,
  visit \href{http://creativecommons.org/licenses/by-sa/3.0/}
  {\color{Purple4}\texttt{http://creativecommons.org/licenses/by-sa/3.0/}}
  or send a letter to Creative Commons, 559 Nathan Abbott Way,
  Stanford, California 94305, USA.

  \begin{description}[Under the following conditions:]
  \tiny
  \item[You are free:] %
    \begin{itemize}
      \tiny
    \item \textbf{to Share} --- to copy, distribute, and transmit the work
    \item \textbf{to Remix} --- to adapt the work
    \item to make commercial use of the work
    \end{itemize}
  \item[Under the following conditions:] %
    \begin{itemize}
      \tiny
    \item \textbf{Attribution} -- You must attribute the work in the manner
      specified by the author or licensor (but not in any way that
      suggests that they endorse you or your use of the work).
    \item \textbf{Share Alike} -- If you alter, transform, or build upon this
      work, you may distribute the resulting work only under the same,
      similar or a compatible license.
    \end{itemize}
  \item[With the understanidng that:] 
    \begin{itemize}
      \tiny
    \item \textbf{Waiver} -- Any of the above conditions can be waived
      if you get permission from the copyright holder.
    \item \textbf{Public Domain} -- Where the work or any of its
      elements is in the public domain under applicable law, that
      status is in no way affected by the license.
    \item \textbf{Other Rights} -- In no way are any of the following
      rights affected by the license:
      \begin{itemize}
      \tiny
      \item Your fair dealing or fair use rights, or other
        applicable copyright exceptions and limitations;
      \item The author's moral rights;
      \item Rights other persons may have either in the work itself
        or in how the work is used, such as publicity or privacy
        rights.
      \end{itemize}
    \item \textbf{Notice} -- For any reuse or distribution, you must
      make clear to others the license terms of this work.
    \end{itemize}
  \end{description}

  This is a human-readable summary of the Legal Code (the full
  license).


\end{frame}

%%% Local Variables:
%%% mode: latex
%%% End:


\begin{frame}
  \frametitle{The nucleotides}
  \begin{columns}[T]
    \begin{column}{0.5\linewidth}
      {\tiny
        \input{nucleotides/adenosinemonophosphate.chemfig}}
    \end{column}
    \begin{column}{0.5\linewidth}
      {\tiny 
        \input{nucleotides/guanisinemonophosphate.chemfig}}
    \end{column}
  \end{columns}

  \bigskip

  \begin{columns}[T]
    \begin{column}{0.5\linewidth}
      {\tiny
        \input{nucleotides/thymidinemonophosphate.chemfig}
      }     
    \end{column}
    \begin{column}{0.5\linewidth}
      {\tiny
        \input{nucleotides/cytodinemonophosphate.chemfig}
      }      
    \end{column}
  \end{columns}

  \begin{textblock*}{0.2\linewidth}(4.5\TPHorizModule,3.0\TPVertModule)%
    Adenisine
  \end{textblock*}
  \begin{textblock*}{0.2\linewidth}(17\TPHorizModule,7.0\TPVertModule)%
    Guanisine
  \end{textblock*}
  \begin{textblock*}{0.2\linewidth}(2\TPHorizModule,12.0\TPVertModule)%
    Thymidine
  \end{textblock*}
  \begin{textblock*}{0.2\linewidth}(13\TPHorizModule,13\TPVertModule)%
    Cytodine
  \end{textblock*}
  \begin{textblock*}{0.2\linewidth}(8.75\TPHorizModule,9\TPVertModule)%
    \textbf{Purines}
  \end{textblock*}
  \begin{textblock*}{0.2\linewidth}(8.5\TPHorizModule,18\TPVertModule)%
    \textbf{Pyridines}
  \end{textblock*}
  
\end{frame}


\begin{frame}
  \frametitle{Hg and thymidine}
  \chemfig{%
    % nitrogenous base begins
    C-[:330]C=[:30,,,,-]C(-[:90,,,,dash bond]Hg\qm)%
    -[:330,,,,-]N(-[:270,,,,-]C(%
    -[:210,,,,-]@{2}N(-[:270,,,,dash bond]@{1}\underline{Hg}\qm)%
    -[:150,,,,-]@{3}C(=[:210,,,,-]O)%
    -[:90,,,,-]C)=[:330,,,,-]O(-[:330,,,,dash bond]\widetilde{Hg}\qm))%
    % sugar begins
    -[:30]C-[:84]C(-[:150,,,,dash bond]\overline{Hg}\qm)-[:12]C(-[:66,,,1]OH^{\color{white}2})%
    -[:300]C(-[:228]O(-[:285,,,,dash bond]\overline{Hg}\qm)%
    % bridge sugar to phosphate
    -[:156]C)-[:354]C-[:294]O-[:354]%
    % phosphate begins
    P(-[:54]\mcfright{O}{^{\mcfminus}})%
    (-[:294]\mcfright{O}{^{\mcfminus}}
    (-[:294,,,,dash bond]\widetilde{Hg}\qm))%
    =[:354]O%
    \namebond{2}{1}{\footnotesize $a$}
    \namebond{2}{3}{\footnotesize $b$}
    \arclabel{0.5cm}{1}{2}{3}{\footnotesize$\varphi$}
  }
\end{frame}


\end{document}

%%% Local Variables:
%%% mode: latex
%%% TeX-master: t
%%% TeX-parse-self: t
%%% TeX-auto-save: t
%%% TeX-auto-untabify: t
%%% TeX-PDF-mode: t
%%% End:
