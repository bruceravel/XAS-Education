\documentclass[10pt, xcolor=x11names, compress]{beamer}
%\documentclass[10pt, xcolor=x11names, compress, handout]{beamer}

\usetheme{progressbar}
%\usecolortheme[named=Purple4]{structure}
\progressbaroptions{headline=sections,titlepage=normal,frametitle=normal}

\setbeamertemplate{navigation symbols}{}

\usepackage{iwona} 

\usepackage{alltt}
\usepackage{amsmath,amsfonts, amssymb, amscd}
\usepackage{hyperref}
\usepackage{setspace}
\usepackage{wasysym}
\usepackage{ulem}

\usepackage{calc}
\usepackage[overlay,absolute]{textpos}
\TPGrid[5mm,5mm]{20}{20}


\renewcommand{\Re}{\operatorname{Re}}
\renewcommand{\Im}{\operatorname{Im}}
\newcommand{\debye}{\operatorname{debye}}

\newcommand{\chik}{$\chi(k)$}
\newcommand{\chir}{$|\tilde{\chi}(R)|$}


\newcommand{\file}[1]{{\color{Firebrick4}\texttt{`#1'}}}
\newcommand{\multiple}{{\color{Orange3}\textsl{multiple}}}


\newcommand{\atoms}  {{\color{DarkOrchid4}\textsc{atoms}}}
\newcommand{\feff}   {{\color{DarkOrchid4}\textsc{feff}}}
\newcommand{\ifeffit}{{\color{DarkOrchid4}\textsc{ifeffit}}}
\newcommand{\athena} {{\color{DarkOrchid4}\textsc{athena}}}
\newcommand{\artemis}{{\color{DarkOrchid4}\textsc{artemis}}}

\renewenvironment<>{center}
{\begin{actionenv}#1\begin{originalcenter}}
{\end{originalcenter}\end{actionenv}}

\hyphenation{microbeam}
\hyphenation{beyond}

\mode<presentation>
%\mode<beamer>

\title{Your Second EXAFS Lecture}%
\subtitle{In which we explore the ins and outs of absorption
  spectroscopy, learn of some the ways to interpret our data, and
  prepare for the rest of this XAFS course}

\author{Bruce Ravel}
\institute[NIST]{Synchrotron Science Group, Materials Measurement Science Division\\%
  Materials Measurement Laboratory\\%
  National Institute of Standards and Technology\\%
  \&\\%
  Beamline for Materials Measurements\\%
  National Synchrotron Light Source II\\~}



\date[Diamond2011]{EXAFS Data Analysis workshop 2011\\
  Diamond Light Source\\November 14--17, 2011}

\begin{document}
\begin{frame}
  \titlepage
\end{frame}

\begin{frame}
  \frametitle{Copyright}
  \tiny

  This document is copyright \copyright\ 2010-2015 Bruce Ravel.

  \begin{center}
    \includegraphics[width=1.0cm]{cc-by-sa.png}
  \end{center}

  This work is licensed under the Creative Commons
  Attribution-ShareAlike License.  To view a copy of this license,
  visit \href{http://creativecommons.org/licenses/by-sa/3.0/}
  {\color{Purple4}\texttt{http://creativecommons.org/licenses/by-sa/3.0/}}
  or send a letter to Creative Commons, 559 Nathan Abbott Way,
  Stanford, California 94305, USA.

  \begin{description}[Under the following conditions:]
  \tiny
  \item[You are free:] %
    \begin{itemize}
      \tiny
    \item \textbf{to Share} --- to copy, distribute, and transmit the work
    \item \textbf{to Remix} --- to adapt the work
    \item to make commercial use of the work
    \end{itemize}
  \item[Under the following conditions:] %
    \begin{itemize}
      \tiny
    \item \textbf{Attribution} -- You must attribute the work in the manner
      specified by the author or licensor (but not in any way that
      suggests that they endorse you or your use of the work).
    \item \textbf{Share Alike} -- If you alter, transform, or build upon this
      work, you may distribute the resulting work only under the same,
      similar or a compatible license.
    \end{itemize}
  \item[With the understanidng that:] 
    \begin{itemize}
      \tiny
    \item \textbf{Waiver} -- Any of the above conditions can be waived
      if you get permission from the copyright holder.
    \item \textbf{Public Domain} -- Where the work or any of its
      elements is in the public domain under applicable law, that
      status is in no way affected by the license.
    \item \textbf{Other Rights} -- In no way are any of the following
      rights affected by the license:
      \begin{itemize}
      \tiny
      \item Your fair dealing or fair use rights, or other
        applicable copyright exceptions and limitations;
      \item The author's moral rights;
      \item Rights other persons may have either in the work itself
        or in how the work is used, such as publicity or privacy
        rights.
      \end{itemize}
    \item \textbf{Notice} -- For any reuse or distribution, you must
      make clear to others the license terms of this work.
    \end{itemize}
  \end{description}

  This is a human-readable summary of the Legal Code (the full
  license).


\end{frame}

%%% Local Variables:
%%% mode: latex
%%% End:


\begin{frame}
  \frametitle{Abstract}
  This lecture is the introductory lecture to a short course on the
  technique of X-ray Absorption Spectroscopy and the anaylsis of XAS
  data.  This is not a ground-level introduction for the complete
  beginner.  We will not derive the EXAFS equation nor explore the
  fundamental physics of the interaction between the photon and the
  absorbing atom.  I assume that you have already been to the
  synchrotron and measured some XAS data.

  \medskip

  The lecture will set the stage for what is to come in this short
  course.  We will motivate the importance of studying XAS in depth,
  even for those who will never make the practice of XAS their primary
  occupation.  We will see an overview of many of the ways to
  interpret and analyze your XAS data.  Finally, we look ahead to the
  end of the course to consider how good experimental practice and
  attention to statistics benefit every XAS practitioner.
\end{frame}

\begin{frame}
  \frametitle{Acknowledgements}
  \begin{columns}[T]
    \begin{column}{0.9\linewidth}
      \begin{itemize}
        \footnotesize
      \item Matt Newville, the author of \textsc{ifeffit}, without
        which \textsc{athena} and \textsc{artemis} would not exist.
      \item Shelly Kelly and Scott Calvin, old friends, relentless
        finders of software bugs, and long-time co-conspirators in
        this XAS eductaion gig.
      \item My teachers, Ed Stern and John Rehr, who have done so much
        for the XAS community and who instilled in me a love of the
        discipline of XAS.
      \item My most wonderful boss, Dan Fischer, for letting me duck
        out of work to do things like this.
      \item The folks who make the great software I use to write my codes:
        \href{http://www.perl.org}{\color{Blue4}Perl},
        \href{http://wxperl.sourceforge.net/}{\color{Blue4}wxPerl},
        \href{http://www.gnu.org/software/emacs/}{\color{Blue4}Emacs},
        \href{http://ecb.sourceforge.net}{\color{Blue4}The Emacs Code Browser},
        \href{http://git-scm.com/}{\color{Blue4}Git},
        \href{http://github.com/}{\color{Blue4}GitHub}.
      \item The folks who make the great software used to write this talk:
        \href{http://tug.ctan.org}{\color{Blue4}\LaTeX},
        \href{http://latex-beamer.sourceforge.net}{\color{Blue4}Beamer},
        \href{http://www.gimp.org}{\color{Blue4}The Gimp},
        \href{http://www.gnuplot.info}{\color{Blue4}Gnuplot}.
      \item Paul Quinn and Diamond for the gracious invitation.
      \item And especially, \textit{all of you}.  It is astonishing and deeply
        flattering that so many people line up to hear what I have to
        say.
      \end{itemize}
    \end{column}
    \begin{column}{0.1\linewidth}
      \includegraphics[width=\linewidth]{mugs/matt.jpg}

      \includegraphics[width=\linewidth]{mugs/shelly.jpg}

      \includegraphics[width=\linewidth]{mugs/scott.jpg}

      \includegraphics[width=\linewidth]{mugs/ed.jpg}

      \includegraphics[width=\linewidth]{mugs/john.jpg}
    \end{column}
  \end{columns}
\end{frame}

\section{Why we measure XAS}

\begin{frame}
  \frametitle{The measurement goals of an XAS experiment}

  Somehow the wiggles in the XAS data tell us something about the
  atomic and electronic structure of the material measured.

  \medskip

  \begin{columns}
    \begin{column}{0.125\linewidth}
      ~
    \end{column}
    \begin{column}{0.45\linewidth}
      \includegraphics[width=\linewidth]{images/eto_mu.png}
    \end{column}
    \begin{column}{0.30\linewidth}
      \includegraphics[width=\linewidth]{../ATEA/mds/eto.png}
    \end{column}
    \begin{column}{0.125\linewidth}
      ~
    \end{column}
  \end{columns}

  \medskip

  \begin{description}
  \item[Valence] the charge state of the absorber
  \item[Species] what kinds of atoms surround the absorber
  \item[Number] how many of those atoms there are
  \item[Distance] how far away they are
  \item[Disorder] how they are distributed due to thermal motion and
    structural disorder
  \end{description}
\end{frame}

\begin{frame}
  \frametitle{What can XAS be measured on}
  Well ... just about anything and with most elements of the periodic
  table

  \begin{columns}[T]
    \begin{column}{0.5\linewidth}
      \begin{exampleblock}{No assumption of symmetry or periodicity}
        \begin{itemize}
        \item Crystals
        \item Liquids or amorphous/highly disordered solids
        \item Mixed phases
        \item Thin films and engineered materials
        \item Surface sorbed species
        \item Organic and organometallic species
        \item Quasicrystals
        \item and on and on
        \end{itemize}
      \end{exampleblock}
    \end{column}
    \begin{column}{0.5\linewidth}
      \begin{alertblock}{Beamline choice and
          sample preparation matter}
        \begin{itemize}
        \item The beamline covers the absorber edge energy
        \item Sample is homogeneous
        \item Sample is not too thick, not too thin
        \item Sample is properly contained
        \end{itemize}
      \end{alertblock}
    \end{column}
  \end{columns}

\end{frame}

\begin{frame}
  \frametitle{Who uses XAS}
  Look to your left and right.  If you study catalysis, you may be
  sitting next to a biologist.  If you are a materials scientist, you
  might be sitting next to a geologist.

  \begin{block}{This fall,$^\ast$ just at my beamline NSLS X23A2, the users include}
    \begin{itemize}
    \item 2 groups from the microelectronics industry
    \item 2 groups of electrochemists
    \item a group working on fuel cell catalysts
    \item a group working on photocathode materials
    \item a group working on nuclear waste containment
    \item a group working on materials for nuclear reactor vessels
    \end{itemize}
  \end{block}

  to say nothing of the biologists, enironmental scientists,
  geologists, and chemical scientists visiting the other beamlines at
  NSLS at the same time.

  \begin{textblock*}{0.5\linewidth}(0pt,20\TPVertModule) 
    \tiny
    $^\ast$September-December 2011
  \end{textblock*}
\end{frame}

\section{How we measure XAS}

\begin{frame}
  \frametitle{Sometimes XAS is easy...}
  \begin{columns}
    \begin{column}{0.36\linewidth}
      \includegraphics[width=\linewidth]{images/gesb_mu.png}

      \includegraphics[width=\linewidth]{images/gesb_chik.png}      

      \includegraphics[width=\linewidth]{images/gesb_chir.png}
    \end{column}
    \begin{column}{0.60\linewidth}
      Here we have a \textit{single scan} of the Ge K edge of a 50\,nm
      film of Ge/Sb alloy on silica measured at glancing angle on a
      beamline (NSLS X23A2) without focusing optics and using a
      Si(311) crystal.
    \end{column}
  \end{columns}
  \begin{textblock*}{0.4\linewidth}(0.6\linewidth,19.0\TPVertModule)%
    \tiny%
    These data are courtesy of Joseph Wasington and Eric
    Joseph (IBM Research)
  \end{textblock*}
\end{frame}

\begin{frame}
  \frametitle{Sometimes XAS is hard...}
  \begin{columns}
    \begin{column}{0.36\linewidth}
      \includegraphics[width=\linewidth]{images/hg_mu_stack.png}

      \includegraphics[width=\linewidth]{images/hg_chik.png}      

      \includegraphics[width=\linewidth]{../ATEA/info/hgdna_chir.png}
    \end{column}
    \begin{column}{0.60\linewidth}
      Here are 42 scans taken over the course of 22 hours at APS 20BM
      (one of the top XAS beamlines in North America) on a sample with
      3\,mM Hg bound to an engineered DNA complex.

      \medskip

      Each scan is quite noisy, but the central limit theorem
      \textit{always} works.  We were able to use the resulting
      $\tilde\chi(R)$ data to good effect.
    \end{column}
  \end{columns}
  \begin{textblock*}{0.4\linewidth}(0.6\linewidth,18.0\TPVertModule)%
    \tiny%
    B.\ Ravel, et al., \textit{EXAFS studies of catalytic DNA sensors
      for mercury contamination of water}, Radiation Physics and
    Chemistry \textbf{78}:10 (2009) pp\ S75-S79.
    \href{http://dx.doi.org/10.1016/j.radphyschem.2009.05.024}
    {\color{Blue4}\texttt{DOI:10.1016/j.radphyschem.2009.05.024}}
  \end{textblock*}
\end{frame}

\begin{frame}
  \frametitle{In any case...}
  Regardless of how easy of difficult an experiment is, we have to
  know how to
  \begin{itemize}
  \item<1-> Evaluate the statistical quality of our data
  \item<2-> Recognize both statistical and systematic errors in our data
    and understand how to address each kind of error
  \item<3-> Know when to stop measuring a sample -- I mean this both in
    the sense of knowing how far above the edge to measure and knowing
    how many repetitions to measure
  \item<4-> Know how to process our data for further analysis
  \end{itemize}
\end{frame}

\begin{frame}
  \frametitle{Sometimes we elaborate experiments}
  \begin{block}{}
    New technologies and modern, 3$^{\mathrm{rd}}$ generation
    synchrotrons offer intriguing new experimental possibilties.
  \end{block}

  \bigskip

  \begin{alertblock}{}
    In each case, one of the results of these elaborate experiments is
    an XAS spectrum.
  \end{alertblock}
\end{frame}

\begin{frame}
  \frametitle{Imaging and $\mu$XAS}
  \begin{columns}[T]
    \begin{column}{0.4\linewidth}
      \includegraphics[width=\linewidth]{images/xrfmap.png}
    \end{column}
    \begin{column}{0.6\linewidth}
      Here is an extraordinary XRF map of a metal hyperaccumulating
      plant that also forms star-shaped, inorganic nodules on its
      leaves.  
      
      \smallskip

      \begin{overlayarea}{\linewidth}{9ex}
        \only<2>{\alert{While the image is itself a great result, we
            end up measuring XAS spectra with the microbeam.}}
      \end{overlayarea}

      \begin{overlayarea}{\linewidth}{9ex}
        \begin{center}
          \includegraphics<2>[width=0.8\linewidth]{images/xrfxas.png}
        \end{center}
      \end{overlayarea}

    \end{column}
  \end{columns}
  \begin{textblock*}{0.4\linewidth}(0pt,18.5\TPVertModule)%
    \tiny%
    R.\ Tappero, et al., \textit{Hyperaccumulator Alyssum murale
      relies on a different metal storage mechanism for cobalt than
      for nickel}, New Phytologist \textbf{175}:4
    (2007) pp\ 641-654
    \href{http://dx.doi.org/10.1111/j.1469-8137.2007.02134.x}
    {\color{Blue4}\texttt{DOI:10.1016/10.1111/j.1469-8137.2007.02134.x}}
  \end{textblock*}
\end{frame}

\begin{frame}
  \frametitle{qXAS and mountains of data}

  \begin{columns}[T]
    \begin{column}{0.55\linewidth}
      \includegraphics[width=0.8\linewidth]{images/ede.png}
    \end{column}
    \begin{column}{0.42\linewidth}
      \includegraphics[width=0.8\linewidth]{images/qxas_mono.png}
    \end{column}
  \end{columns}

  \begin{columns}
    \begin{column}{0.5\linewidth}
      \begin{overlayarea}{\linewidth}{24ex}
        Energy dispersive XAS and quick-XAS are two ways of doing
        time-resolved XAS with 10\,ms to 10\,s time resolution.  Both
        approaches require specially- equipped beamlines and both
        focus on the dynamics of the system. \only<2>{\alert{At then
            end of the day, these are XAS spectra.}}
      \end{overlayarea}
    \end{column}
    \begin{column}{0.5\linewidth}
      \includegraphics[width=0.8\linewidth]{images/qxas.png}
    \end{column}
  \end{columns}

  \begin{textblock*}{0.7\linewidth}(0pt,19.5\TPVertModule)%
    \tiny%
    Data from W.A.\ Caliebe et al., \textit{HASYLAB Annual Report} (2006) pp.\
    283-284; EDE schematic from SPring-8 press release, 30 April,
    2009; QXAS schematic from SLS SuperXAS beamline webpage.
  \end{textblock*}
\end{frame}

\begin{frame}
  \frametitle{DAFS}

  With coordinated motion between monochromator and goniometer,
  diffraction anomalous fine structure measures the height of a
  diffraction peak with respect to energy through the resonant energy
  of an atom in the crystal.
  \begin{columns}[T]
    \begin{column}{0.5\linewidth}
      \includegraphics[width=0.8\linewidth]{images/dafs.png}
    \end{column}
    \begin{column}{0.5\linewidth}
      \includegraphics[width=0.8\linewidth]{images/dafschik.png}      
    \end{column}
  \end{columns}
  \begin{overlayarea}{\linewidth}{6ex}
    \only<2>{\alert{In the end, we extract a site-specific $\chi(k)$
        function which is analyzed like normal EXAFS.}}
  \end{overlayarea}

  \begin{textblock*}{0.6\linewidth}(0pt,19.5\TPVertModule)%
    \tiny%
    B.\ Ravel et al., \textit{X-ray-absorption edge separation using
      diffraction anomalous fine structure}, Phys.\ Rev.\ B \textbf{60}
    (1999) pp\ 778-785.
    \href{http://dx.doi.org/10.1103/PhysRevB.60.778}
    {\color{Blue4}\texttt{DOI:10.1103/PhysRevB.60.778}}
  \end{textblock*}
\end{frame}

\begin{frame}
  \frametitle{NIXS}

  Here is data from a non-resonant inelastic scattering data on
  CaZrTi$_2$O$_7$ from 20ID at APS.
  \begin{columns}
    \begin{column}{0.4\linewidth}
      \includegraphics[width=\linewidth]{images/lerix_sm.jpg}
    \end{column}
    \begin{column}{0.6\linewidth}
      \includegraphics[width=0.75\linewidth]{images/nixs.png}

      \includegraphics[width=0.75\linewidth]{images/TiL23.png}
    \end{column}
  \end{columns}

  \begin{overlayarea}{\linewidth}{2ex}
    \only<2>{\alert{Again, a XANES spectrum comes from this elaborate
        experiment.}}
  \end{overlayarea}

\end{frame}


\section{How we use XAS}

\begin{frame}
  \frametitle{Fingerprinting}
  The simplest way of using XAS is to simply identify chemical species
  in a sample of unknown composition.

  \begin{columns}
    \begin{column}{0.5\linewidth}
      \includegraphics[width=\linewidth]{images/fe_stan.png}
    \end{column}
    \begin{column}{0.25\linewidth}
      \begin{enumerate}
      \item \only<1|handout:0>{Ferrihydrite?}\only<2|handout:1>{\color{Green4}Ferrihydrite}
      \item \only<1|handout:0>{Iron metal?}\only<2|handout:1>{\color{Blue4}Iron metal}
      \item \only<1|handout:0>{Iron pyrite?}\only<2|handout:1>{\color{Red3}Iron pyrite}
      \item \only<1|handout:0>{Hematite?}\only<2|handout:1>{\color{Purple4}Hematite}
      \end{enumerate}
    \end{column}
    \begin{column}{0.15\linewidth}
    \end{column}
  \end{columns}
  
  \bigskip

  \begin{overlayarea}{\linewidth}{4ex}
    \only<2|handout:1>{Compare the signal from your \textit{( dirt /
        catalyst / paint chip / animal tissue / whatever )} with
      the standards to identify the dominant species.}
  \end{overlayarea}
\end{frame}

\begin{frame}
  \frametitle{Some vocabulary}
  Words commonly used to describe specific parts of the XANES spectrum.

  \begin{columns}[T]
    \begin{column}{0.5\linewidth}
      \includegraphics[width=\linewidth]{../i2x/images/rams/ramsdellite.png}

      \small

      \begin{description}[edge]
      \item[{\color{red}pre-edge}] Small (possibly large, certainly
        meaningful!) features between the Fermi energy and the
        threshold
      \item[{\color{DarkOrchid2}edge}] The main rising part of XAS spectrum
      \item[{\color{Green4}near-edge}] Characteristic features above the edge
      \end{description}
    \end{column}    
    \begin{column}{0.5\linewidth}
      \includegraphics[width=\linewidth]{../i2x/images/ltno/ltno.png}

      \begin{description}[wh]
      \item[{\color{Blue3}white line}] Large, prominent peak just
        above the edge, particularly in L or M edge spectra
      \end{description}

      \bigskip

      \begin{block}{}
        The EXAFS, then, is the data beyond the
        {\color{Green4}near-edge}.
      \end{block}
    \end{column}    
  \end{columns}
  
\end{frame}

\begin{frame}
  \frametitle{XANES Interpretation}
  Fingerprinting, though useful, is a qualitative tool.  There are a
  variety of quantitive tools available for interpreting XANES data.
  \begin{enumerate}
  \item Positioning
  \item Peak fitting
  \item Linear combination fitting
  \item Principle components analysis
  \item Theory
  \end{enumerate}
\end{frame}

\begin{frame}
  \frametitle{XANES: Positioning}
  \begin{columns}[T]
    \begin{column}{0.5\linewidth}
      \small%
      U$^{6+}$ is partially reduced by \textit{G. sulfurreducens}
      under a variety of conditions.  By measuring the edge positions
      and comparing to U$^{6+}$ and U$^{4+}$ standards,
      the amount of reduction is quatified between 24\% and 88\%.

      \includegraphics[width=0.8\linewidth]{images/u46.png}      
    \end{column}
    \begin{column}{0.5\linewidth}
      \small%
      Farges et al. showed that Ti clusters by valence when the
      position of the largest {\color{red}pre-edge} is plotted against
      peak position in energy.

      \includegraphics[width=0.8\linewidth]{../selfabs/images/farges.png}      
    \end{column}
  \end{columns}
  \begin{textblock*}{0.5\linewidth}(0pt,18.75\TPVertModule)%
    \tiny%
    B.H.\ Jeon et el., \textit{Microbial Reduction of U(VI) at the
      Solid-Water Interface}, Environ. Sci. Technol. \textbf{38}(21)
    pp.~5649-5655. (2004), \href{http://dx.doi.org/10.1021/es0496120}
    {\color{Blue4}\texttt{DOI:10.1021/es0496120}}
  \end{textblock*}
  \begin{textblock*}{0.4\linewidth}(0.6\linewidth,18.75\TPVertModule)
    \tiny%
    F.~Farges, G.E.~Brown, Jr., and J.J.~Rehr, PRB \textbf{56}:4,
    (1997) p.\ 1809.
    \href{http://dx.doi.org/10.1103/PhysRevB.56.1809}
    {\color{Blue4}\texttt{DOI:10.1103/PhysRevB.56.1809}}
  \end{textblock*}

\end{frame}

\begin{frame}
  \frametitle{XANES: Peak fiting}
  titanate perovskites?
\end{frame}

\begin{frame}
  \frametitle{XANES: Linear Combination Fitting}
  In this example, XAS is measured as a function of time as a gold
  chloride is reduced to metallic gold in the presence of sulfurous
  biomass.  At an intermediate time step, the spectrum is understood
  as a linear combination of the initial state
  ({\color{Green4}Au$^{3+}$Cl}), the final state ({\color{Purple4}Au
    metal}), and an intermediate state ({\color{Orange2}some Au$^{1+}$
    sulfide species}).
  \begin{columns}
    \begin{column}{0.5\linewidth}
      \includegraphics[width=\linewidth]{../iiss/xas/aucl_lcf.png}
    \end{column}
    \begin{column}{0.5\linewidth}
      \includegraphics[width=\linewidth]{../iiss/xas/aucl_results.png}
    \end{column}
  \end{columns}
  \begin{flushright}
    \scriptsize (There will be an entire lecture on this topic tomorrow.)
  \end{flushright}
  \begin{textblock*}{0.5\linewidth}(0pt,18.75\TPVertModule) 
    \tiny
    M. Lengke et el., \textit{Mechanisms of Gold Bioaccumulation by
      Filamentous Cyanobacteria from Gold(III)-Chloride Complex},
    Environ. Sci. Technol. \textbf{40}(20) p.~6304-6309. (2006),
    \href{http://dx.doi.org/10.1021/es061040r}
    {\color{Blue4}\texttt{DOI:10.1021/es061040r}}
  \end{textblock*}
\end{frame}

\begin{frame}
  \frametitle{XANES: Principle Components Analysis}
  \small%
  PCA is a bit of linear algebra which breaks down an ensemble of
  related data into abstract components.
  \begin{columns}[T]
    \begin{column}{0.33\linewidth}
      \includegraphics[width=\linewidth]{images/aucyano_time_series.png}
    \end{column}
    \begin{column}{0.33\linewidth}
      \includegraphics[width=\linewidth]{images/pca_all.png}      
    \end{column}
    \begin{column}{0.33\linewidth}
      \includegraphics[width=\linewidth]{images/pca_2-8.png}            
    \end{column}
  \end{columns}

  \medskip

  The components can then be used to try to construct a standard as a
  test to see whether that standard is present in the ensemble.

  \medskip

  \begin{columns}
    \begin{column}{0.33\linewidth}
      The number of species represented in the ensemble is related to
      the number of statistically significant components.
    \end{column}
    \begin{column}{0.33\linewidth}
      \includegraphics[width=\linewidth]{images/pca_tt_in}
    \end{column}
    \begin{column}{0.33\linewidth}
      \includegraphics[width=\linewidth]{images/pca_tt_out}
    \end{column}
  \end{columns}
  \begin{textblock*}{0.5\linewidth}(0pt,18.75\TPVertModule)%
    \tiny%
    The data are from M. Lengke et el.,
    Environ. Sci. Technol. \textbf{40}(20) p.~6304-6309. (2006),
    \href{http://dx.doi.org/10.1021/es061040r}
    {\color{Blue4}\texttt{DOI:10.1021/es061040r}}.  That paper did not
      include any PCA analysis.
  \end{textblock*}
\end{frame}

\begin{frame}
  \frametitle{XANES: Theory}
  \begin{columns}
    \begin{column}{0.5\linewidth}
      Forward simulation
    \end{column}
    \begin{column}{0.5\linewidth}
      MXAN and FitIt
    \end{column}
  \end{columns}
\end{frame}

\begin{frame}
  \frametitle{EXAFS Analysis}

\end{frame}

\begin{frame}
  \frametitle{EXAFS analysis can be simple...}
  
\end{frame}

\begin{frame}
  \frametitle{EXAFS analysis can be sophisticated...}
  
\end{frame}

\begin{frame}
  \frametitle{EXAFS analysis can be quite elaborate...}
  Scott's mixed ferrites
\end{frame}

\section{How we understand XAS}

\begin{frame}
  \frametitle{Real space multiple scattering}
  
\end{frame}

\begin{frame}
  \frametitle{A bit of math}
  
\end{frame}

\begin{frame}
  \frametitle{The EXAFS equation}
  
\end{frame}

\begin{frame}
  \frametitle{XAS out of the vacuum}
  
\end{frame}

\begin{frame}
  \frametitle{XAS and statistics}
\end{frame}

\begin{frame}
  \frametitle{Ready, set, go!}
  \begin{alertblock}{}
    \begin{center}
      \LARGE \alert{Shall we begin?}
    \end{center}

  \end{alertblock}
\end{frame}
\end{document}
