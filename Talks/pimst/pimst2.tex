% see https://tex.stackexchange.com/a/74828
\PassOptionsToPackage{x11names}{xcolor}
\documentclass[10pt, xcolor=x11names, compress]{beamer}
%\documentclass[10pt, xcolor=x11names, compress, handout]{beamer}
\usetheme{progressbar}
%\usecolortheme[named=Purple4]{structure}
\progressbaroptions{headline=sections,titlepage=normal,frametitle=normal}

\setbeamertemplate{navigation symbols}{}

\usepackage{iwona} 

\usepackage{alltt}
\usepackage{amsmath,amsfonts, amssymb, amscd}
\usepackage{hyperref}
\usepackage{setspace}
\usepackage{wasysym}
\usepackage{ulem}
\usepackage{xspace}

\usepackage{calc}
\usepackage[overlay,absolute]{textpos}
\TPGrid[5mm,5mm]{20}{20}


\usepackage[x11names]{xcolor}

\usepackage{marvosym}
\newcommand{\homepagesymbol}{{\Large\ComputerMouse~}}%      {{\Large\marvosymbol{205}}~}


\renewcommand{\Re}{\operatorname{Re}}
\renewcommand{\Im}{\operatorname{Im}}
\newcommand{\debye}{\operatorname{debye}}

\newcommand{\chik}{$\chi(k)$}
\newcommand{\chir}{$|\tilde{\chi}(R)|$}


\newcommand{\file}[1]{{\color{Firebrick4}\texttt{`#1'}}}
\newcommand{\multiple}{{\color{Orange3}\textsl{multiple}}}

\definecolor{Programs}{rgb}{0.0,0.2,0.0}
\newcommand{\atoms}    {{\color{Programs}\textsc{atoms}}\xspace}
\newcommand{\feff}     {{\color{Programs}\textsc{feff}}\xspace}
\newcommand{\feffex}   {{\color{Programs}\textsc{feff85exafs}}\xspace}
\newcommand{\feffsix}  {{\color{Programs}\textsc{feff6}}\xspace}
\newcommand{\feffeight}{{\color{Programs}\textsc{feff8}}\xspace}
\newcommand{\feffnine} {{\color{Programs}\textsc{feff9}}\xspace}
\newcommand{\ifeffit}  {{\color{Programs}\textsc{ifeffit}}\xspace}
\newcommand{\larch}    {{\color{Programs}\textsc{larch}}\xspace}
\newcommand{\athena}   {{\color{Programs}\textsc{athena}}\xspace}
\newcommand{\artemis}  {{\color{Programs}\textsc{artemis}}\xspace}

\renewenvironment<>{center}
{\begin{actionenv}#1\begin{originalcenter}}
{\end{originalcenter}\end{actionenv}}

\definecolor{guessp}   {rgb}{0.64,0.00,0.64}
\newcommand{\guessp}   {{\color{guessp}guess}}
\definecolor{defp}     {rgb}{0.00,0.55,0.00}
\newcommand{\defp}     {{\color{defp}def}}
\definecolor{setp}     {rgb}{0,0,0}
\newcommand{\setp}     {{\color{setp}set}}
\definecolor{lguessp}  {rgb}{0.24,0.11,0.56}
\newcommand{\lguessp}  {{\color{lguessp}lguess}}
\definecolor{skipp}    {rgb}{0.70,0.70,0.70}
\newcommand{\skipp}    {{\color{skipp}skip}}
\definecolor{restrainp}{rgb}{0.80,0.61,0.11}
\newcommand{\restrainp}{{\color{restrainp}restrain}}
\definecolor{afterp}   {rgb}{0.29,0.44,0.55}
\newcommand{\afterp}   {{\color{afterp}after}}
\definecolor{penaltyp} {rgb}{0.55,0.35,0.17}
\newcommand{\penaltyp} {{\color{penaltyp}penalty}}
\definecolor{mergep}   {rgb}{0.93,0.00,0.00}
\newcommand{\mergep}   {{\color{mergep}merge}}

\hyphenation{EXAFS}

\newcommand{\exafsequation}[1][\small]{%
  {#1
    \begin{align*}
      \chi(k,\Gamma) =& 
      { \frac{{\color{Red4}(N_\Gamma S_0^2)}{\color{Blue4}F_\Gamma(k)}
                        e^{-2{\color{Red4}\sigma_\Gamma^2}k^2}
                        e^{-2R_\Gamma/{\color{Blue4}\lambda(k)}}
                        }
                        {2\,kR_\Gamma^2} }
      \sin{(2kR_\Gamma + {\color{Blue4}\Phi_\Gamma(k)})} \\
      \chi_{\mathrm{theory}}(k) =& \sum\limits_{\Gamma}\chi(k,\Gamma)\\
      R_\Gamma =& \> {\color{Blue4}R_{0,\Gamma}} +
      {\color{Red4}\Delta R_\Gamma} \\
      k =& \sqrt{2m_e(E_0 - {\color{Red4}\Delta E_0})/\hbar^2} 
           \approx \sqrt{(E_0 - {\color{Red4}\Delta E_0})/3.81}
    \end{align*}}
}


%% inline enumeration, see
%% http://tex.stackexchange.com/questions/94478/beamer-inline-itemize-and-enumeration/94521#94521
\newcounter{newenumi}
\setcounter{newenumi}{1}
\newcommand{\inlineenum}{%
 {%
 \setcounter{enumi}{\thenewenumi}%
 \leavevmode\usebeamertemplate{enumerate  item}
 \stepcounter{newenumi}
 \setcounter{enumi}{0}
 }
}
\newcommand{\resetinlineenum}{\setcounter{newenumi}{1}}

\usepackage{xparse}
\definecolor{bngray}{rgb}{0.5,0.5,0.5}
\NewDocumentEnvironment{bottomnote}{O{0.5}O{19.5}}%[0.85][19.5]
{\begin{textblock*}{#1\linewidth}(0pt,#2\TPVertModule)%
   \tiny\begin{color}{bngray}}%
{\end{color}\end{textblock*}}

\NewDocumentCommand{\cornerlogo}{m}
{\begin{textblock*}{0.08\linewidth}(17.2\TPHorizModule,0\TPVertModule)%
    \includegraphics[width=2cm]{#1}%
  \end{textblock*}} 
\NewDocumentCommand{\smcornerlogo}{m}
{\begin{textblock*}{0.08\linewidth}(19.5\TPHorizModule,0\TPVertModule)%
    \includegraphics[width=7mm]{#1}%
  \end{textblock*}} 


\DeclareDocumentCommand{\doiref}{mO{Blue4}}%
{\href{https://doi.org/#1}{\color{#2}{\ComputerMouse~}DOI:~#1}}

\DeclareDocumentCommand{\inlinelogo}{m}%
{\raisebox{-.2\height}{\includegraphics[width=5mm]{#1}}\xspace}

\DeclareDocumentCommand{\titlepageurl}{mO{Blue4}}%
{~\\[2ex]{\footnotesize\href{#1}{\color{#2}%
    {\ComputerMouse\,}PDF of this talk: #1}}}

\newtheorem{conclusion}[theorem]{Conclusion}
\newtheorem{notethis}[theorem]{Note}

\newcommand{\eto}{EuTiO$_3$}
\newcommand{\pto}{PbTiO$_3$}
\newcommand{\pgt}{Pb$_{1-x}$Ge$_x$Te}
\newcommand{\lsco}{La$_{1-x}$Sr$_x$CuO$_4$}
\newcommand{\abc}{AgBr$_{1-x}$Cl$_x$}
\newcommand{\nicn}{$\big[$Ni(CN)$_4\big]$$^{2-}$}
\newcommand{\knicn}{K$_2$Ni(CN)$_4$}

% \newcommand{\GMS}{\mathbb{G}}
% \newcommand{\Gnot}{\mathsf{G^0}}
% \newcommand{\tmat}{\mathsf{t}}
% \newcommand{\boldr}{\boldsymbol{r}}

\include{scatter}

\mode<presentation>

\title{A Practical Introduction to Multiple Scattering Theory}
\include{author}
%\date[APS EXAFS 07]{2007 APS EXAFS Summer School\\July 23-27, 2007\\~}
\date{\today}

\begin{document}
\maketitle

\begin{frame}
  \frametitle{Copyright}
  \tiny

  This document is copyright \copyright\ 2010-2015 Bruce Ravel.

  \begin{center}
    \includegraphics[width=1.0cm]{cc-by-sa.png}
  \end{center}

  This work is licensed under the Creative Commons
  Attribution-ShareAlike License.  To view a copy of this license,
  visit \href{http://creativecommons.org/licenses/by-sa/3.0/}
  {\color{Purple2}\texttt{http://creativecommons.org/licenses/by-sa/3.0/}}
  or send a letter to Creative Commons, 559 Nathan Abbott Way,
  Stanford, California 94305, USA.

  \begin{description}[Under the following conditions:]
  \tiny
  \item[You are free:] %
    \begin{itemize}
      \tiny
    \item \textbf{to Share} --- to copy, distribute, and transmit the work
    \item \textbf{to Remix} --- to adapt the work
    \item to make commercial use of the work
    \end{itemize}
  \item[Under the following conditions:] %
    \begin{itemize}
      \tiny
    \item \textbf{Attribution} -- You must attribute the work in the manner
      specified by the author or licensor (but not in any way that
      suggests that they endorse you or your use of the work).
    \item \textbf{Share Alike} -- If you alter, transform, or build upon this
      work, you may distribute the resulting work only under the same,
      similar or a compatible license.
    \end{itemize}
  \item[With the understanidng that:] 
    \begin{itemize}
      \tiny
    \item \textbf{Waiver} -- Any of the above conditions can be waived
      if you get permission from the copyright holder.
    \item \textbf{Public Domain} -- Where the work or any of its
      elements is in the public domain under applicable law, that
      status is in no way affected by the license.
    \item \textbf{Other Rights} -- In no way are any of the following
      rights affected by the license:
      \begin{itemize}
      \tiny
      \item Your fair dealing or fair use rights, or other
        applicable copyright exceptions and limitations;
      \item The author's moral rights;
      \item Rights other persons may have either in the work itself
        or in how the work is used, such as publicity or privacy
        rights.
      \end{itemize}
    \item \textbf{Notice} -- For any reuse or distribution, you must
      make clear to others the license terms of this work.
    \end{itemize}
  \end{description}

  This is a human-readable summary of the Legal Code (the full
  license).


\end{frame}

%%% Local Variables:
%%% mode: latex
%%% End:

\begin{frame}
  \frametitle{Acknowledgements}
  \footnotesize
  \begin{tabular}{cc}
    \begin{minipage}{0.1\linewidth}
      \includegraphics[width=\linewidth]{mugs/matt.jpg}
    \end{minipage}&
    \begin{minipage}{0.7\linewidth}
      Matt Newville, author of {\ifeffit} and author of a
      presentation which covers similar material to this talk.
    \end{minipage} \\
    \begin{minipage}{0.1\linewidth}
      \includegraphics[width=\linewidth]{mugs/john.jpg}
    \end{minipage}&
    \begin{minipage}{0.7\linewidth}
      John Rehr and his group, authors of {\feff}.
    \end{minipage} \\
    \begin{minipage}{0.1\linewidth}
      \includegraphics[width=\linewidth]{mugs/ed.jpg}
    \end{minipage}&
    \begin{minipage}{0.7\linewidth}
      Ed Stern, for teaching us all so well and for getting all this XAS
      stuff started in the first place.
    \end{minipage}
  \end{tabular}

  \medskip

  \begin{itemize}
    \footnotesize
  \item The many users of my software: without years of feedback and
    encouragement, my codes would suck way more than they do
  \item The folks who make the great software I use to write my codes:
    \href{http://www.perl.org}{\color{Blue4}Perl},
    \href{http://wxperl.sourceforge.net/}{\color{Blue4}wxPerl},
    \href{http://www.gnu.org/software/emacs/}{\color{Blue4}Emacs},
    \href{http://ecb.sourceforge.net}{\color{Blue4}The Emacs Code Browser},
    \href{http://git-scm.com/}{\color{Blue4}Git},
    \href{http://github.com/}{\color{Blue4}GitHub}
  \item The folks who make the great software used to write this talk:
    \href{http://tug.ctan.org}{\color{Blue4}\LaTeX},
    \href{http://latex-beamer.sourceforge.net}{\color{Blue4}Beamer},
    \href{http://avogadro.sourceforge.net}{\color{Blue4}Avogadro},
    \href{http://inkscape.net}{\color{Blue4}Inkscape},
    \href{http://www.gimp.org}{\color{Blue4}The Gimp},
    \href{http://www.gnuplot.info}{\color{Blue4}Gnuplot}
  \end{itemize}
\end{frame}

\section{Introduction}

\begin{frame}
  \frametitle{What I hope you take away from this talk}
  \begin{itemize}
  \item A broad outline of multiple scattering theory with enough
    background to talk with a theorist
  \item An understanding of how multiple scattering theory is used to
    interpret \textbf{XANES} spectra
  \item An understanding of how multiple scattering theory is used to
    analyze \textbf{EXAFS} spectra
  \item Some ideas about how to incorporate multiple scattering theory
    in your research
  \end{itemize}
\end{frame}

\begin{frame}
  \frametitle{This talk is about Feff}

  There are many approaches to spectroscopy theory out there,
  including multiplets, band structure, and finite difference methods.

  \bigskip

  \begin{exampleblock}{This talk is about Feff}
    \textsc{feff} is a real-space, multiple scattering code.
  \end{exampleblock}
  
  \bigskip

  \begin{itemize}
  \item A conceptual summary and simple physical interpretation of
    what ``real-space multiple scattering'' means.
  \item How RSMS is used to make XANES calculations.
  \item How RSMS is used in fitting EXAFS data.
  \end{itemize}
\end{frame}

\begin{frame}
  \frametitle{XAS Data}
  \begin{columns}<1->
    \begin{column}{0.35\linewidth}
      \includegraphics[width=\linewidth]{images/FeS2_mu.png}
    \end{column}
    \begin{column}{0.65\linewidth}
      We measure the {\color{Blue4}XAS data} and find the
      {\color{Red3}background function}
      \begin{equation}
        \mu(E) = \mu_0(E)\cdot\big(1+\chi(E)\big)
        \notag
      \end{equation}
    \end{column}
  \end{columns}
  \begin{columns}<2->
    \begin{column}{0.35\linewidth}
      \includegraphics[width=\linewidth]{images/FeS2_chik.png}
    \end{column}
    \begin{column}{0.65\linewidth}
      We subtract the background, $\mu_0(E)$, to isolate the ``fine
      structure'' $\chi(k)$.  

      {\scriptsize (Remember, \alert{EXAFS} $\equiv$ \alert{E}xtended
        \alert{X}-ray \alert{A}bsorption \underline{\alert{F}ine
        \alert{S}tructure}.)}
    \end{column}
  \end{columns}
  \begin{columns}<3>
    \begin{column}{0.35\linewidth}
      \includegraphics[width=\linewidth]{images/FeS2_chir.png}
    \end{column}
    \begin{column}{0.65\linewidth}
      We Fourier transform $\chi(k)$ and use \alert{multiple scattering
        theory} to understand the local structure.
    \end{column}
  \end{columns}
\end{frame}

\section{Real-space multiple scattering}

\begin{frame}
  \frametitle<1| handout:1>{A simple picture of X-ray absorption} 
  \frametitle<2| handout:2>{X-ray absorption in condensed matter}

  \begin{overlayarea}{\linewidth}{6ex}
    \only<1| handout:1>{An incident x-ray of energy $E$ is absorbed,
      destroying a core electron of binding energy $E_0$ and emitting
      a photo-electron with kinetic energy $(E-E_0)$. The core state
      is eventually filled, ejecting a fluorescent x-ray or an Auger
      electron.  }%
    \only<2| handout:2>{The ejected photo-electron can scatter from
      neighboring atoms. $R$ has some relationship to $\lambda$ and
      there is a phase shift associated with the scattering event.
      Thus the outgoing and scattered waves interfere.  }%
  \end{overlayarea}

  \vskip 20pt

  \begin{columns}[T]
    \begin{column}{0.5\linewidth}
      \only<1| handout:1>{\includegraphics[width=\linewidth]{bare_atom.png}}
      \only<2| handout:2>{\includegraphics[width=\linewidth]{with_scattering.png}}
    \end{column}
    \begin{column}{0.5\linewidth}
      \begin{center}
        \only<1| handout:1>{
          An empty final state is required.\\
          {\alert{No available state, \\
              no absorption!}}\\
          When the incident x-ray energy is larger than the binding
          energy, there is a sharp increase in absorption.
        }
        \only<2| handout:2>{
          The scattering of the photo-electron wave function interferes
          with itself.\\[4ex]
          $\mu(E)$ depends on the density of states with energy
          $(E-E_0)$ at the absorbing atom.
        }
      \end{center}
    \end{column}
  \end{columns}

  \vskip 10pt

  \begin{overlayarea}{\linewidth}{3ex}
    \only<1| handout:1>{For an isolated atom, $\mu(E)$ has a sharp step at the
      core-level binding energy and is a smooth function of energy
      above the edge.}%
    \only<2| handout:2>{This interference \alert{at the absorbing atom} will vary
      with energy, causing the oscillations in $\mu(E)$.
    }%
  \end{overlayarea}
  \begin{bottomnote}[0.7][20] 
    Image from \href{https://millenia.cars.aps.anl.gov/xafs/APS_2005/Newville_Intro.pdf}{\color{LightBlue4}{\ComputerMouse~}Matt Newville}
  \end{bottomnote}
\end{frame}

\include{firstprinciples}


\newcommand{\FeLeft}{0.60\linewidth}
\newcommand{\FeRight}{0.38\linewidth}
\newcommand{\FeCluster}{0.65\linewidth}
\newcommand{\FePlot}{0.78\linewidth}

\begin{frame}
  \frametitle<1|handout:1>{Iron metal: 1$^{\mathrm{st}}$ path, 1 shell }
  \frametitle<2|handout:2>{Iron metal: 2$^{\mathrm{nd}}$ path, 2 shells }
  \frametitle<3|handout:3>{Iron metal: 3$^{\mathrm{rd}}$ path, 1 shell }
  \frametitle<4|handout:4>{Iron metal: 4$^{\mathrm{th}}$ path, 2 shells }
  \frametitle<5|handout:5>{Iron metal: 5$^{\mathrm{th}}$ path, 3 shells }
  \frametitle<6|handout:6>{Iron metal: 8$^{\mathrm{th}}$ path, 4 shells }

  \begin{columns}[T]
    \begin{column}{0.60\linewidth}
      \centering\includegraphics<1|handout:1>[width=\FeCluster]{images/path1_diagram}
      \centering\includegraphics<2|handout:2>[width=\FeCluster]{images/path2_diagram}
      \centering\includegraphics<3|handout:3>[width=\FeCluster]{images/path3_diagram}
      \centering\includegraphics<4|handout:4>[width=\FeCluster]{images/path4_diagram}
      \centering\includegraphics<5|handout:5>[width=\FeCluster]{images/path5_diagram}
      \centering\includegraphics<6|handout:6>[width=\FeCluster]{images/path8_diagram}
      \begin{enumerate}
      \item %
        \only<1|handout:1>{The first path is much, but not all, of the first peak in {\chir}.}
        \only<2|handout:2>{The second path overlaps the first in {\chir}.}
        \only<3|handout:3>{This path contributes little to {\chir}.}
        \only<4|handout:4>{This path contributes little to {\chir}.}
        \only<5|handout:5>{This 3$^{\mathrm{rd}}$ shell SS path contributes most
          of the spectral weight to the second peak of {\chir}.}
        \only<6|handout:6>{The 4$^{\mathrm{th}}$ shell SS path contributes to
          the third peak in {\chir}.}\\
        {\color{Firebrick3}Degeneracy = \only<1|handout:1>{8}\only<2|handout:2>{6}\only<3|handout:3>{24}\only<4|handout:4>{48}\only<5|handout:5>{12}\only<6|handout:6>{24}}
      \item %
        \only<1|handout:1>{The first shell XANES calculation shows little of the
          structure.}
        \only<2|handout:2>{The XANES calculation begins to show the structure of
          the spectrum.}
        \only<3|handout:3>{The contribution from this path and all higher order
          paths scattering among these atoms is in the first shell
          XANES calculation.}
        \only<4|handout:4>{The contribution from this path and all higher order
          paths scattering among these the first two shells is in the
          second shell XANES calculation.}
        \only<5|handout:5>{The first peak after the edge in the XANES is
          sharpened considerably by the addition of this shell.}
        \only<6|handout:6>{Including this shell in the XANES calculation broadens
          the peak above the edge somewhat.  It also introduces the
          second shoulder.}
      \end{enumerate}
    \end{column}
    \begin{column}{0.38\linewidth}
      \centering
      \includegraphics<1|handout:1>[width=\FePlot]{images/path1}
      \includegraphics<2|handout:2>[width=\FePlot]{images/path2}
      \includegraphics<3|handout:3>[width=\FePlot]{images/path3}
      \includegraphics<4|handout:4>[width=\FePlot]{images/path4}
      \includegraphics<5|handout:5>[width=\FePlot]{images/path5}
      \includegraphics<6|handout:6>[width=\FePlot]{images/path8}

      \centering\color{Green4}\texttt{`feff000{\only<1>{1}\only<2>{2}\only<3>{3}\only<4>{4}\only<5>{5}\only<6>{8}}.dat'}
      %\only<2>{\color{Green4}\texttt{`feff0002.dat'}}
      %\only<3>{\color{Green4}\texttt{`feff0003.dat'}}
      %\only<4>{\color{Green4}\texttt{`feff0004.dat'}}
      %\only<5>{\color{Green4}\texttt{`feff0005.dat'}}
      %\only<6>{\color{Green4}\texttt{`feff0008.dat'}}

      \bigskip

      \includegraphics<1|handout:1>[width=\FePlot]{images/xanes_shell1}
      \includegraphics<2|handout:2>[width=\FePlot]{images/xanes_shell2}
      \includegraphics<3|handout:3>[width=\FePlot]{images/xanes_shell1}
      \includegraphics<4|handout:4>[width=\FePlot]{images/xanes_shell2}
      \includegraphics<5|handout:5>[width=\FePlot]{images/xanes_shell3}
      \includegraphics<6|handout:6>[width=\FePlot]{images/xanes_shell4}

      \centering{\color{blue}XANES}

      \bigskip

      ~

      \bigskip

      ~

    \end{column}
  \end{columns}
\end{frame}



\begin{frame}
  \frametitle{Iron metal: 10$^{\mathrm{th}}$ path + MS, 5 shells}
  \begin{columns}[T]
    \begin{column}{0.05\linewidth}
      ~
    \end{column}
    \begin{column}{0.4\linewidth}
      \begin{center}
        \includegraphics[width=0.75\linewidth]{images/path10}

        5$^{\mathrm{th}}$ shell EXAFS: Magnitude\\[2ex]

        \includegraphics[width=0.75\linewidth]{images/path10_re}

        5$^{\mathrm{th}}$ shell EXAFS: Real part
      \end{center}
    \end{column}
    \begin{column}{0.1\linewidth}
      ~
    \end{column}
    \begin{column}{0.4\linewidth}
      \begin{block}{Convergence}
        \includegraphics[width=\linewidth]{images/xanes_convergence}
      \end{block}
      There are several MS geometries with the same path length as the
      5$^{\mathrm{th}}$ shell SS path.  Some are \emph{bigger} than
      the SS path!
    \end{column}
    \begin{column}{0.05\linewidth}
      ~
    \end{column}
  \end{columns}
\end{frame}


% \section[XANES]{XANES Calculations}
% \subsection[Convergence of XANES]{\pto: Convergence of Full Multiple Scattering}

% \include{size}
% \subsection[Interesting XANES Problems]{Solving interesting XANES problems}
% \include{hardxanes}

%%\section[EXAFS]{Using Multiple Scattering in EXAFS}

\section{EXAFS equation}

%\againframe{fgr}

\begin{frame}
  \frametitle{Fermi's Golden Rule revisited}
  The absorption is the dipole mediated transition from the initial
  state of the deep-core electron to its final state:
  \begin{equation}
    \mu(E) \sim \big|\langle f| \mathcal{H}|i\rangle\big|^2
    \notag
  \end{equation}
  \begin{description}
  \item[The initial state $|i\rangle$] This is the deep core, atomic
    state which is unaffected by the surroundings
  \item[The excitation $\mathcal{H}$] The dipole operator, i.e.\ the
    incident photon
  \item[The final state $|f\rangle$] This high-lying or continuum
    state \alert{is} affected by the surroundings
  \end{description}
  \begin{block}{Consider $|f\rangle = |f_0+\Delta f\rangle$}
    \begin{itemize}
    \item $|f_0\rangle$ is the final state in the presence of the
      surrounding atoms but \alert{without} any scattering of the
      photoelectron
    \item $\Delta f$is the purturbation to the final state cause by
      the scattering of the photoelectron from the surrounding atoms
    \end{itemize}
  \end{block}
  \begin{bottomnote}[0.6][19]
    The discussion on the following 8 pages is inspired by Matt
    Newville's at
    \href{http://xafs.org/Tutorials?action=AttachFile&do=view&target=Newville_Intro.pdf}
    {\color{LightBlue4}\texttt{http://xafs.org/Tutorials?action=AttachFile\&do=view\&target=Newville\_Intro.pdf}}
  \end{bottomnote}
\end{frame}


\begin{frame}
  \frametitle{The fine structure}
  With $|f\rangle = |f_0+\Delta f\rangle$
  \begin{align}
    \mu(E) \sim&\, \big|\langle \alert{f}| \mathcal{H}|i\rangle\big|^2
    \notag\\
    \sim&\, \big|\langle \alert{f_0}| \mathcal{H}|i\rangle\big|^2
    \big[
    1 + A(E)
    \big|\langle \alert{\Delta f}| \mathcal{H}|i\rangle\big|
    + C.C.
    \big]\notag\\
    \intertext{Remember that}
    \mu(E) =&\, \mu_0(E) \cdot (1+\chi(E))\notag\\
    \intertext{Therefore}
    \chi(E) \sim&\,\Big(\big|\langle \alert{\Delta f}|
    \mathcal{H}|i\rangle\big| + C.C.\Big)\notag
  \end{align}

  \begin{conclusion}
    The XAS fine structure, $\chi(E)$, is caused by the scattering from
    the neighboring atoms.
  \end{conclusion}

  \begin{bottomnote}[0.6][19.25]
    $A(E)$ contains a bunch of stuff having nothing to do with the
    scattering. $A(E) = \langle
    i|\mathcal{H}|\alert{f_0}\rangle^\ast / \big|\langle
    \alert{f_0}| \mathcal{H}|i\rangle\big|^2$
  \end{bottomnote}
\end{frame}

\begin{frame}
  \frametitle{Heuristic derivation of the EXAFS equation}
  \begin{columns}
    \begin{column}{0.5\linewidth}
      The photoelectron:
      \begin{itemize}
      \item propagates as a spherical wave from absorber to scatterer
      \item scatters from the neighbor
      \item propagates as a spherical wave from scatterer to absorber
      \end{itemize}
    \end{column}
    \begin{column}{0.5\linewidth}
      \includegraphics[width=\linewidth]{images/circles.png}
    \end{column}
  \end{columns}
  
  \bigskip

  Energy and photoelectron wavenumber are related by
  \begin{equation}
    k = \sqrt{2m_e(E-E_0)/\hbar^2} \simeq \alert{\sqrt{(E-E_0)/3.81}}\notag
  \end{equation}

  So, in terms of $k$
  \begin{equation}
    \chi(k) \sim \frac{e^{ikr}}{kr} \cdot
    2kF(k)e^{\phi(k)} \cdot
    \frac{e^{ikr}}{kr} + C.C.
    \notag
  \end{equation}
\end{frame}

\begin{frame}
  \frametitle{The EXAFS equation in its simplest form}
  We can now simplify the equation to
  \begin{equation}
    \chi(k) \sim \frac{F(k)}{2kR^2}\sin\big(2kR+\phi(k)\big)
    \notag
  \end{equation}
  This describes the signal from a single atom at a distance $R$.

  \medskip

  If we consider the contribution from $N$ atoms at distance $R$
  (i.e.\ a ``shell'' of atoms):
  \begin{equation}
    \chi(k) \sim \frac{NF(k)}{2kR^2}\sin\big(2kR+\phi(k)\big)
    \notag
  \end{equation}
  On the following pages, we consider
  \begin{enumerate}
  \item the shapes of $F(k)$ and $\phi(k)$
  \item the amplitude reduction term $S_0^2$
  \item the mean free path term $\lambda$
  \item disorder via the mean square displacement term $\sigma^2$
  \end{enumerate}
\end{frame}

\begin{frame}
  \frametitle{The complex photoelectron scattering factor}
  The scattering function, $F(k)$ and $\phi(k)$ give EXAFS its
  sensitivity to atomic species.
  \begin{equation}
    \chi(k) \sim \frac{N\alert{F(k)}}{2kR^2}\sin\big(2kR+\alert{\phi(k)}\big)
    \notag
  \end{equation}
  \begin{columns}
    \begin{column}{0.5\linewidth}
      \begin{center}
        Magnitude

        \includegraphics[width=\linewidth]{images/f_eff.png}
      \end{center}
    \end{column}
    \begin{column}{0.5\linewidth}
      \begin{center}
        Phase

        \includegraphics[width=\linewidth]{images/phi_eff.png}
      \end{center}
    \end{column}
  \end{columns}
  Examining the magnitude explains why the signal from light elements
  does not extend much beyond 10\,\AA$^{-1}$.
\end{frame}

\begin{frame}
  \frametitle{The amplitude reduction factor}

  When the core electron is ejected from it's deep-core state, the
  remaining electrons relax:
  \begin{equation}
    S_0^2 = \big|\langle \Phi_f^{N-1}  | \Phi_i^{N-1}  \rangle
    \big|^2    \notag
  \end{equation}
  where $|\Phi^{N-1}\rangle$ is the state of all remaining electrons
  before ($i$) or after ($f$) the excitation.
  \begin{equation}
    \chi(k) \sim \frac{N\alert{S_0^2}F(k)}{2kR^2}\sin\big(2kR+\phi(k)\big)
    \notag
  \end{equation}
  In practice, $0.7\lesssim S_0^2<1.0$, but note that $N$ and $S_0^2$
  are completely correlated!
  \begin{bottomnote}[0.6][19.25]
    G.G.\ Li, F.\ Bridges, \& C.H.\ Booth, Phys. Rev. B \textbf{52}
    (1995) 6332--6348
    \doiref{10.1103/PhysRevB.52.6332}[LightBlue4]
  \end{bottomnote}
\end{frame}

\begin{frame}
  \frametitle{The mean free path}
  The photoelecton may scatter \textit{inelastically} and fail to
  ``return'' to the absorber (lose coherence with the core-hole).

  \medskip

  \begin{columns}
    \begin{column}{0.5\linewidth}
      We consider this by replacing the photoelecton spherical wave
      with a damped spherical wave:
      $\frac{e^{ikr}e^{-r/\lambda(k)}}{kr}$

      \medskip

      Here is {\feff}'s calculation of the mean free path in copper
      metal.
    \end{column}
    \begin{column}{0.5\linewidth}
      \includegraphics[width=\linewidth]{images/mfp.png}
    \end{column}
  \end{columns}
  \begin{equation}
    \chi(k) \sim \frac{NS_0^2F(k)}{2kR^2}\sin\big(2kR+\phi(k)\big)
    \alert{e^{-2R/\lambda(k)}}
    \notag
  \end{equation}
  \begin{notethis}
    $\frac{e^{-2R/\lambda(k)}}{R^2}$ is what makes EXAFS a \alert{local}
    structure probe.
  \end{notethis}
\end{frame}

\begin{frame}
  \frametitle{The mean square displacement (disorder)}
  Even in a highly ordered crystal -- like an FCC metal -- the atoms
  are never actually on their lattice positions.  \textit{Thermal
    motion} (i.e.\ phonons) distribute atoms around their nominal
  positions such that
  \begin{equation}
    \sigma_{i,j}^2 = \langle r_{i,j}-\overline{r_{i,j}} \rangle^2 > 0
    \notag
  \end{equation}
  This behaves some like the crystallographic Debye-Waller factor:
  \begin{exampleblock}{The standard EXAFS equation}
    \begin{equation}
      \chi(k) = \frac{NS_0^2F(k)}{2kR^2}\sin\big(2kR+\phi(k)\big)
      \alert{e^{-2k^2\sigma^2}}e^{-2r/\lambda(k)}
      \notag
    \end{equation}
  \end{exampleblock}

  \begin{bottomnote}[0.8][17.5]
    {\scriptsize
      One can also consider higher moments of the distribution,
      $\sigma^n = \langle r_{i,j}-\overline{r_{i,j}} \rangle^n$.
    }\\[1ex]
    {\tiny%
      See G.\ Bunker, Nucl.\ Inst.\ Methods \textbf{207}:3 (1983) pp.\
      437--444,
      \doiref{10.1016/0167-5087(83)90655-5}[LightBlue4]
    }
  \end{bottomnote}
\end{frame}

\begin{frame}
  \frametitle{Multiple scattering paths}
  The magic of {\feff} is that it expresses the effect of multiple
  scattering events entirely in $F(k)$ and $\phi(k)$:
  \begin{equation}
    \chi(k) = \frac{NS_0^2\alert{F_{eff}(k)}}{2kR^2}
    \sin\big(2kR+\alert{\phi_{eff}(k)}\big)
    e^{-2k^2\sigma^2}e^{-2r/\lambda(k)}
    \notag
  \end{equation}
  That's the same equation!

  \begin{bottomnote}[0.6][19.5]
    S.I.\ Zabinsky et al, Phys. Rev. B \textbf{52} (1995) 2995--3009\\
    \doiref{10.1103/PhysRevB.52.2995}[LightBlue4]
  \end{bottomnote}
\end{frame}

\section[EXAFS]{Using FEFF to Solve EXAFS Problems}

\begin{frame}[fragile]
  \frametitle<1|handout:1>{A Feff6 input file}
  \frametitle<2|handout:2>{A Feff8 input file}
  \begin{columns}[T]
    \begin{column}{0.44\linewidth}
      Here is an example of a \textsc{feff}\only<1>6\only<2>8 input file:

      \vspace*{\stretch{1}}

      \begin{onlyenv}<1|handout:1>
        \begin{block}{}
          \begin{alltt}
            \tiny
 {\color{Green4}TITLE Cobalt sulfide  CoS\_2}

 {\color{Purple2}HOLE} 1 1.0 {\color{Blue4}*  Co K edge (7709.0 eV)

 *         mphase,mpath,mfeff,mchi}
 {\color{SteelBlue2}CONTROL}   1      1     1     1
 {\color{SteelBlue2}PRINT}     1      0     0     0

 {\color{Purple2}RMAX}        6.0


 {\color{Brown4}POTENTIALS}
 {\color{Blue4}*    ipot   Z  element}
        0   27   Co
        1   27   Co
        2   16   S

                  {\color{Blue4}* continued ------>}
          \end{alltt}
        \end{block}
      \end{onlyenv}
      \begin{onlyenv}<2|handout:2>
        \begin{block}{}
          \begin{alltt}
            \tiny
 {\color{Green4}TITLE Cobalt sulfide  CoS\_2}
 {\color{Purple2}EDGE} K
 {\color{Purple2}S02}  1.0

 {\color{Blue4} *    pot    xsph  fms   paths genfmt ff2chi}
 {\color{SteelBlue2}CONTROL}   1      1     1     1     1     1
 {\color{SteelBlue2}PRINT}     1      0     0     0     0     0

 {\color{Purple2}EXCHANGE}   0
 {\color{Purple2}SCF}        4.0
 {\color{Purple2}XANES}      4.0
 {\color{Purple2}FMS}        5.09694  0
 {\color{Purple2}LDOS}      -30   20     0.1
 {\color{Purple2}RPATH}      0.1
 {\color{Blue4}*EXAFS     20}

 {\color{Brown4}POTENTIALS}
 {\color{Blue4}*   ipot  Z  element  l\_scmt  l\_fms  stoi.}
        0   27   Co       2       2       0
        1   27   Co       2       2       4
        2   16   S        2       2       8
                          {\color{Blue4}* continued ------>}
          \end{alltt}
        \end{block}
      \end{onlyenv}
     \end{column}
     %%
     \begin{column}{0.54\linewidth}
      \begin{block}{}
         \begin{alltt}
         \tiny
  {\color{Brown4}ATOMS}   {\color{Blue4}* this list contains 71 atoms
  *   x          y          z     ipot  tag     distance}
     0.00000    0.00000    0.00000  0  Co1     0.00000
     2.14845    0.61305    0.61305  2  S1\_1    2.31678
     0.61305   -2.14845    0.61305  2  S1\_1    2.31678
    -0.61305    0.61305    2.14845  2  S1\_1    2.31678
    -0.61305    2.14845   -0.61305  2  S1\_1    2.31678
    -2.14845   -0.61305   -0.61305  2  S1\_1    2.31678
     0.61305   -0.61305   -2.14845  2  S1\_1    2.31678
    -3.37455    0.61305    0.61305  2  S1\_2    3.48415
     0.61305    3.37455    0.61305  2  S1\_2    3.48415
     0.61305   -0.61305    3.37455  2  S1\_2    3.48415
     3.37455   -0.61305   -0.61305  2  S1\_2    3.48415
    -0.61305   -3.37455   -0.61305  2  S1\_2    3.48415
    -0.61305    0.61305   -3.37455  2  S1\_2    3.48415
    -2.14845   -2.14845    2.14845  2  S1\_3    3.72122
     2.14845    2.14845   -2.14845  2  S1\_3    3.72122
     2.76150    2.76150    0.00000  1  Co1\_1   3.90535
    -2.76150    2.76150    0.00000  1  Co1\_1   3.90535
     2.76150   -2.76150    0.00000  1  Co1\_1   3.90535
    -2.76150   -2.76150    0.00000  1  Co1\_1   3.90535
     2.76150    0.00000    2.76150  1  Co1\_1   3.90535
    -2.76150    0.00000    2.76150  1  Co1\_1   3.90535
     0.00000    2.76150    2.76150  1  Co1\_1   3.90535
  {\color{Blue4}*
  * etc...
  *}
  {\color{Purple2}END}
         \end{alltt}
       \end{block}
     \end{column}
   \end{columns}
\end{frame}


\begin{frame}[fragile]
  \frametitle{Using \textsc{atoms} to prepare the \textsc{feff} input file}
  %%
  \textsc{artemis} includes a tool called \textsc{atoms} that converts
  crystallographic data into a \textsc{feff} input file.

  \medskip

  \begin{columns}[c]
    \begin{column}[c]{0.5\linewidth}
      \begin{center}
        \includegraphics[width=0.7\linewidth]{images/artemis_atoms}
      \end{center}
    \end{column}
    %%
    \begin{column}[c]{0.5\linewidth}
      The input data can be a CIF file or this simple format:
      \begin{block}{}
        \begin{alltt}
          \tiny
 {\color{Green4}title Cobalt sulfide}
 {\color{Green4}title Elliot (1960) J.Chem. Phys. 33(3), 903.}
 {\color{Brown4}space} P a 3
 {\color{Brown4}rmax}=6.0   {\color{Brown4}a}=5.523
 {\color{Brown4}core}=Co
 {\color{Brown4}atoms}
 {\color{Blue4}! At.type   x     y     z      tag}
    Co     0.00000   0.00000   0.00000  Co
    S      0.38900   0.38900   0.38900  S
         \end{alltt}
       \end{block}
     \end{column}
   \end{columns}

   \bigskip

   These data are typically taken from the crystallography literature,
   the \textit{Inorganic Crystal Structure Database}, or from:
   \href{http://cars9.uchicago.edu/\char126newville/adb/search.html}
   {\color{Blue4}\texttt{http://cars9.uchicago.edu/\char126newville/adb/search.html}}
\end{frame}

\begin{frame}[fragile]
  \frametitle{Feff input files for non-crystalline materials}
  \begin{columns}[c]
    \begin{column}{0.6\linewidth}
      There are many sources of structural data about molecules,
      proteins, and other non-crystalline materials. A bit of googling
      turned up this Protein Data Bank File for cisplatin:

      \centering\includegraphics<1>[width=0.4\linewidth]{images/cisplatin}
      \begin{alltt}
        \tiny
ATOM   1 PT1  MOL A  1  -0.142   0.141   7.747  1.00  1.00
ATOM   2 CL2  MOL A  1  -0.135  -2.042   8.092  1.00  1.00
ATOM   3 CL3  MOL A  1   2.064   0.127   7.615  1.00  1.00
ATOM   4  N4  MOL A  1  -0.147   2.166   7.427  1.00  1.00
ATOM   5  N5  MOL A  1  -2.188   0.154   7.870  1.00  1.00
ATOM   6 1H4  MOL A  1   0.793   2.489   7.319  1.00  1.00
ATOM   7 2H4  MOL A  1  -0.570   2.625   8.208  1.00  1.00
ATOM   8 3H4  MOL A  1  -0.668   2.370   6.598  1.00  1.00
ATOM   9 1H5  MOL A  1  -2.464   0.303   8.819  1.00  1.00
ATOM  10 2H5  MOL A  1  -2.546  -0.724   7.552  1.00  1.00
ATOM  11 3H5  MOL A  1  -2.551   0.889   7.298  1.00  1.00
TER
      \end{alltt}
    \end{column}
    %%
    \begin{column}{0.4\linewidth}
      Cut, paste, insert some boilerplate, and voil\'a!
      \begin{block}{}
        \begin{alltt}
          \tiny
 {\color{Green4}TITLE cisplatin}
 {\color{Purple2}HOLE}  4  1.0
 {\color{Purple2}RMAX}  8
 {\color{Brown4}POTENTIALS}
     0   78   Pt
     1   17   Cl
     2    7   N
     3    1   H

 {\color{Brown4}ATOMS}
   -0.142   0.141   7.747   0
   -0.135  -2.042   8.092   1
    2.064   0.127   7.615   1
   -0.147   2.166   7.427   2
   -2.188   0.154   7.870   2
    0.793   2.489   7.319   3
   -0.570   2.625   8.208   3
   -0.668   2.370   6.598   3
   -2.464   0.303   8.819   3
   -2.546  -0.724   7.552   3
   -2.551   0.889   7.298   3
         \end{alltt}
       \end{block}

       ~\\[-5ex]

       {\scriptsize Note that the absorber need not be at
         (0,0,0) and the list need not be in any particular order.}

    \end{column}
  \end{columns}
\end{frame}


%\subsection[Atoms and paths]{Atoms and paths}
%\include{paths}
%\subsection[Good practice]{Using \textsc{feff} well}
%\include{tricks}

\begin{frame}
  \frametitle{Multiple scattering and EXAFS: FeS$_2$}
  \begin{columns}
    \begin{column}{0.5\linewidth}
      \includegraphics[width=\linewidth]{images/FeS2/fes2.png}
      \begin{center}
        {\color{Chocolate3}$\bullet$} = Fe \quad%
        {\color{Gold2}$\bullet$} = S
      \end{center}
    \end{column}
    \begin{column}{0.5\linewidth}
      \includegraphics[width=\linewidth]{images/intrp.png}
    \end{column}
  \end{columns}
\end{frame}

\begin{frame}
  \frametitle{Multiple scattering and EXAFS: SS}
  \begin{columns}
    \begin{column}{0.5\linewidth}
      The first sulfur SS path is from the octahedron surrounding the
      Fe atom.  It provides most of the spectral weight under the
      first peak.

      \medskip

      The next two S and one Fe SS paths overlap between 2.5 and
      3.5\,\AA.
    \end{column}
    \begin{column}{0.5\linewidth}
      \includegraphics[width=\linewidth]{images/fes2_ss.png}
    \end{column}
  \end{columns}
\end{frame}

\begin{frame}
  \frametitle{Multiple scattering and EXAFS: MS}
  \begin{columns}
    \begin{column}{0.5\linewidth}
      The relationship between the EXAFS spectrum and atomic
      structure can be quite complicated due to multiple
      scattering.

      \medskip

      S--S and S--Fe triangles contribute
      significantly between 2.5 and 3.5\,\AA.

      \medskip

      Collinear paths through the absorber involving 1$^{st}$ shell S
      atoms contribute significantly around 3.9\,\AA.
    \end{column}
    \begin{column}{0.5\linewidth}
      \includegraphics[width=\linewidth]{images/fes2_ms.png}
    \end{column}
  \end{columns}
\end{frame}

\section[Resources]{Resources}

\begin{frame}
  \frametitle{Resources}
  \begin{itemize}
  \item Websites
    \begin{itemize}
    \item \scriptsize
      \href{http://xafs.org}{\color{Blue4}\texttt{http://xafs.org}}
      offers tutorials, links to resources, information about upcoming
      workshops, and much more
    \item \scriptsize {\larch} homepage:
      \href{http://xraypy.github.io/xraylarch/}
      {\color{Blue4}\texttt{http://xraypy.github.io/xraylarch/}}
    \item \scriptsize {\ifeffit} mailing list:
      \href{http://cars9.uchicago.edu/mailman/listinfo/ifeffit}
      {\color{Blue4}\texttt{http://cars9.uchicago.edu/mailman/listinfo/ifeffit}}
    \item \scriptsize {\feff} homepage:
      \href{http://feff.phys.washington.edu}
      {\color{Blue4}\texttt{http://feff.phys.washington.edu}}
    \item \scriptsize {\athena} and {\artemis}:
      \href{http://github.com/bruceravel/demeter}
      {\color{Blue4}\texttt{http://github.com/bruceravel/demeter/}}
    \end{itemize}
  \item Journal articles
    \begin{itemize}
    \item \scriptsize The \textsc{feff} reference: Rehr and Albers
      review article: J.J.~Rehr and R.C.~Albers, Rev.\ Mod.\ Phys.\
      \textbf{72}:3 (2000) pp.\ 621--654.  Also see subsequent
      references from Rehr for {\feff}8 and {\feff}9.
    \item \scriptsize Two excellent references on multiple scattering theory:
      J.L.~Beeby, Proc.\ Royal Soc.\ \textbf{A274} (1964) pp.\
      309--317 and \textbf{A279} (1967) pp.\ 82--97.
    \end{itemize}
  \item Other Software
    \begin{itemize}
    \item \scriptsize XANES calculations using Mulitplets:
      \href{http://xafs.org/Software/TtMultiplet}
      {\color{Blue4}\texttt{http://xafs.org/Software/TtMultiplet}}
    \item \scriptsize XANES calculations by finite difference method:
      \href{http://xafs.org/Software/FDMNES}
      {\color{Blue4}\texttt{http://xafs.org/Software/FDMNES}}
    \item \scriptsize Band structure: The work of Eric Shirley
      (\href{http://physics.nist.gov/Divisions/Div844/facilities/theorModel/tmopm.html}
      {\color{Blue4}\tiny \texttt{http://physics.nist.gov/Divisions/Div844/facilities/theorModel/tmopm.html}}) and
      Aleksi Soininen, Helsinki University
    \item \scriptsize XANES fitting: \textsc{FitIt} (\href{http://xafs.org/Software/FitIt}{\color{Blue4}\texttt{http://xafs.org/Software/FitIt}})
      and \textsc{mxan} (PRB \textbf{65} (2002) 174205).
    \end{itemize}
  \end{itemize}
\end{frame}


\end{document}


%%% Local Variables:
%%% TeX-parse-self: t
%%% TeX-auto-save: t
%%% TeX-auto-untabify: t
%%% TeX-PDF-mode: t
%%% End:
