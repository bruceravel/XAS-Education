\documentclass[10pt, xcolor=x11names, compress]{beamer}
%\documentclass[10pt, xcolor=x11names, compress, handout]{beamer}

\usetheme{progressbar}
%\usecolortheme[named=Purple4]{structure}
\progressbaroptions{headline=sections,titlepage=normal,frametitle=normal}

\setbeamertemplate{navigation symbols}{}

\usepackage{iwona} 

\usepackage{alltt}
\usepackage{amsmath,amsfonts, amssymb, amscd}
\usepackage{hyperref}
\usepackage{setspace}
\usepackage{wasysym}
\usepackage{ulem}
\usepackage{xspace}

\usepackage{calc}
\usepackage[overlay,absolute]{textpos}
\TPGrid[5mm,5mm]{20}{20}



\renewcommand{\Re}{\operatorname{Re}}
\renewcommand{\Im}{\operatorname{Im}}
\newcommand{\debye}{\operatorname{debye}}

\newcommand{\chik}{$\chi(k)$}
\newcommand{\chir}{$|\tilde{\chi}(R)|$}


\newcommand{\file}[1]{{\color{Firebrick4}\texttt{`#1'}}}
\newcommand{\multiple}{{\color{Orange3}\textsl{multiple}}}

\definecolor{Programs}{rgb}{0.0,0.2,0.0}
\newcommand{\atoms}    {{\color{Programs}\textsc{atoms}}\xspace}
\newcommand{\feff}     {{\color{Programs}\textsc{feff}}\xspace}
\newcommand{\feffex}   {{\color{Programs}\textsc{feff85exafs}}\xspace}
\newcommand{\feffsix}  {{\color{Programs}\textsc{feff6}}\xspace}
\newcommand{\feffeight}{{\color{Programs}\textsc{feff8}}\xspace}
\newcommand{\feffnine} {{\color{Programs}\textsc{feff9}}\xspace}
\newcommand{\ifeffit}  {{\color{Programs}\textsc{ifeffit}}\xspace}
\newcommand{\athena}   {{\color{Programs}\textsc{athena}}\xspace}
\newcommand{\artemis}  {{\color{Programs}\textsc{artemis}}\xspace}

\renewenvironment<>{center}
{\begin{actionenv}#1\begin{originalcenter}}
{\end{originalcenter}\end{actionenv}}

\definecolor{guessp}   {rgb}{0.64,0.00,0.64}
\newcommand{\guessp}   {{\color{guessp}guess}}
\definecolor{defp}     {rgb}{0.00,0.55,0.00}
\newcommand{\defp}     {{\color{defp}def}}
\definecolor{setp}     {rgb}{0,0,0}
\newcommand{\setp}     {{\color{setp}set}}
\definecolor{lguessp}  {rgb}{0.24,0.11,0.56}
\newcommand{\lguessp}  {{\color{lguessp}lguess}}
\definecolor{skipp}    {rgb}{0.70,0.70,0.70}
\newcommand{\skipp}    {{\color{skipp}skip}}
\definecolor{restrainp}{rgb}{0.80,0.61,0.11}
\newcommand{\restrainp}{{\color{restrainp}restrain}}
\definecolor{afterp}   {rgb}{0.29,0.44,0.55}
\newcommand{\afterp}   {{\color{afterp}after}}
\definecolor{penaltyp} {rgb}{0.55,0.35,0.17}
\newcommand{\penaltyp} {{\color{penaltyp}penalty}}
\definecolor{mergep}   {rgb}{0.93,0.00,0.00}
\newcommand{\mergep}   {{\color{mergep}merge}}

\hyphenation{EXAFS}

%% inline enumeration, see
%% http://tex.stackexchange.com/questions/94478/beamer-inline-itemize-and-enumeration/94521#94521
\newcounter{newenumi}
\setcounter{newenumi}{1}
\newcommand{\inlineenum}{%
 {%
 \setcounter{enumi}{\thenewenumi}%
 \leavevmode\usebeamertemplate{enumerate  item}
 \stepcounter{newenumi}
 \setcounter{enumi}{0}
 }
}
\newcommand{\resetinlineenum}{\setcounter{newenumi}{1}}

\usepackage{xparse}
\definecolor{bngray}{rgb}{0.5,0.5,0.5}
\NewDocumentEnvironment{bottomnote}{O{0.5}O{19.5}}%[0.85][19.5]
{\begin{textblock*}{#1\linewidth}(0pt,#2\TPVertModule)%
   \tiny\begin{color}{bngray}}%
{\end{color}\end{textblock*}}

\NewDocumentCommand{\cornerlogo}{m}
{\begin{textblock*}{0.08\linewidth}(17.2\TPHorizModule,0\TPVertModule)%
    \includegraphics[width=2cm]{#1}%
  \end{textblock*}} 


\DeclareDocumentCommand{\doiref}{m}%
{\href{http://dx.doi.org/#1}{\color{Blue4}DOI:~#1}}


\newtheorem{clt}[theorem]{The Central Limit Theorem}
% \newtheorem{assertion}[theorem]{Assertion}
% \newtheorem{exercise}[theorem]{Exercise for the reader}
% \newtheorem{remember}[theorem]{Always remember}

\mode<presentation>

\title{The Central Limit Thoerem \textit{Always} Works!}%
\subtitle{Statistics, EXAFS, and Knowing when to stop measuring data}

\author{Bruce Ravel}
\institute[NIST]{Synchrotron Science Group, Materials Measurement Science Division\\%
  Materials Measurement Laboratory\\%
  National Institute of Standards and Technology\\%
  \&\\%
  Beamline for Materials Measurements\\%
  National Synchrotron Light Source II\\~}


\date{\today}

\begin{document}

\maketitle

\begin{frame}
  \frametitle{Copyright}
  \tiny

  This document is copyright \copyright\ 2010-2015 Bruce Ravel.

  \begin{center}
    \includegraphics[width=1.0cm]{cc-by-sa.png}
  \end{center}

  This work is licensed under the Creative Commons
  Attribution-ShareAlike License.  To view a copy of this license,
  visit \href{http://creativecommons.org/licenses/by-sa/3.0/}
  {\color{Purple4}\texttt{http://creativecommons.org/licenses/by-sa/3.0/}}
  or send a letter to Creative Commons, 559 Nathan Abbott Way,
  Stanford, California 94305, USA.

  \begin{description}[Under the following conditions:]
  \tiny
  \item[You are free:] %
    \begin{itemize}
      \tiny
    \item \textbf{to Share} --- to copy, distribute, and transmit the work
    \item \textbf{to Remix} --- to adapt the work
    \item to make commercial use of the work
    \end{itemize}
  \item[Under the following conditions:] %
    \begin{itemize}
      \tiny
    \item \textbf{Attribution} -- You must attribute the work in the manner
      specified by the author or licensor (but not in any way that
      suggests that they endorse you or your use of the work).
    \item \textbf{Share Alike} -- If you alter, transform, or build upon this
      work, you may distribute the resulting work only under the same,
      similar or a compatible license.
    \end{itemize}
  \item[With the understanidng that:] 
    \begin{itemize}
      \tiny
    \item \textbf{Waiver} -- Any of the above conditions can be waived
      if you get permission from the copyright holder.
    \item \textbf{Public Domain} -- Where the work or any of its
      elements is in the public domain under applicable law, that
      status is in no way affected by the license.
    \item \textbf{Other Rights} -- In no way are any of the following
      rights affected by the license:
      \begin{itemize}
      \tiny
      \item Your fair dealing or fair use rights, or other
        applicable copyright exceptions and limitations;
      \item The author's moral rights;
      \item Rights other persons may have either in the work itself
        or in how the work is used, such as publicity or privacy
        rights.
      \end{itemize}
    \item \textbf{Notice} -- For any reuse or distribution, you must
      make clear to others the license terms of this work.
    \end{itemize}
  \end{description}

  This is a human-readable summary of the Legal Code (the full
  license).


\end{frame}

%%% Local Variables:
%%% mode: latex
%%% End:


\section{Introduction}

\begin{frame}
  \frametitle{On a good day...} 

  \small
  ... we measure {\color{Green4}\textit{beautiful}} data.  This is
  the merge of 5 scans on a 50\,nm film of GeSb on silica, at the Ge
  edge and measured in fluorescence at NSLS X23A2.\\~

  \begin{columns}
    \begin{column}{0.5\linewidth}
      \begin{center}
        \includegraphics[width=0.8\linewidth]{../ATEA/info/gesb_chik.png}
      \end{center}
    \end{column}
    \begin{column}{0.5\linewidth}
      \begin{center}
        \includegraphics[width=0.8\linewidth]{../ATEA/info/gesb_chir.png}
      \end{center}
    \end{column}
  \end{columns}
  Here, I show a Fourier transform window of [3\,:\,13] and I suggest a
  fitting range of [1.7\,:\,4.7].  Applying the Nyquist criterion:
  \begin{equation}
    \notag N_{idp} \approx \frac{2\Delta k\Delta R}{\pi} \approx \alert{19}
  \end{equation}

  ~\\[-7ex]
  ~

  \begin{exampleblock}{}
    \begin{center}
      Did I really need to measure 5 scans?  Could I have stopped
      after a single scan?
    \end{center}
  \end{exampleblock}
  \begin{textblock*}{0.4\linewidth}(0pt,19.25\TPVertModule)%
    \tiny%
    These data are courtesy of Joseph Washington and Eric
    Joseph (IBM Research)
  \end{textblock*}
\end{frame}

\begin{frame}
  \frametitle{On all the rest of the days...}
  \small
  Sometimes, we have less-than-beautiful data.  This is the merge of 42
  scans on a solution containing 3\,mM of Hg bound to a synthetic DNA
  complex, measured in fluorescence at APS 20BM.
  \begin{columns}
    \begin{column}{0.5\linewidth}
      \begin{center}
        \includegraphics[width=0.8\linewidth]{../ATEA/info/hgdna_chik.png}
      \end{center}
    \end{column}
    \begin{column}{0.5\linewidth}
      \begin{center}
        \includegraphics[width=0.8\linewidth]{../ATEA/info/hgdna_chir.png}
      \end{center}
    \end{column}
  \end{columns}
  Here, I show a Fourier transform window of [2\,:\,8.8] and I suggest a
  fitting range of [1\,:\,3].  Applying the Nyquist criterion:
  \begin{equation}
    \notag N_{idp} \approx \frac{2\Delta k\Delta R}{\pi} \approx \alert{8}
  \end{equation}

  ~\\[-7ex]
  ~

  \begin{exampleblock}{}
    \begin{center}
      Many real research problems are more like this.  Why were 42
      scans measured?  Was that too many?  Not enough?  How can we know?
    \end{center}
  \end{exampleblock}

  \begin{textblock*}{0.7\linewidth}(0pt,19.0\TPVertModule)%
    \tiny%
    B.\ Ravel, et al., \textit{EXAFS studies of catalytic DNA sensors
      for mercury contamination of water}, Radiation Physics and
    Chemistry \textbf{78}:10 (2009) pp\ S75-S79.
    \href{http://dx.doi.org/10.1016/j.radphyschem.2009.05.024}
    {\color{Blue4}\texttt{DOI:10.1016/j.radphyschem.2009.05.024}}
  \end{textblock*}
\end{frame}


\begin{frame}
  %\frametitle{The Central Limit Theorem}
  \begin{clt}
    Given certain conditions, the mean of a sufficiently large number
    of independent random variables, each with finite mean and
    variance, will be approximately normally distributed.
  \end{clt}

  \bigskip

  In the context of an EXAFS measurement, the CLT tells us that, when a
  noisy spectrum that is dominated by statistical noise, the spectral
  noise will be distributed normally about its mean.

  \bigskip

  If we measure enough repititions of data dominated by statistical
  noise and merge the data by computing the arithmetic mean at every
  energy point, the data will converge to the mean.

  \begin{block}{In short...}
    With patience, ugly data becomes beautiful.
  \end{block}
\end{frame}


\section{Practical matters}

\begin{frame}
  \frametitle{The most basic rule of thumb}
  Before making a measurement, you have no idea what the data will
  look like.  You \textbf{cannot} know how many repititions will be
  required before examining the first scan.
  \begin{description}
  \item[One scan?] Never$^*$ measure a single scan.  How would you
    know if something went wrong with the measurement?
  \item[Two scans?] What if the two repititions are different?  How do
    know which one is right?
  \item[Three scans?] There you go!  Now you can know which on is
    right.  \alert{Always plan on at least three repititions.}
  \end{description}
  \begin{textblock*}{0.7\linewidth}(0pt,17.5\TPVertModule)%
    \footnotesize%
    $^*$ Did I just say ``never''?  Yikes!  \textit{Never} say
    ``never''!\\Why, on the very next page I am going to show examples
    where single scans were measured.
  \end{textblock*}
\end{frame}

\begin{frame}
  \frametitle{Rules of thumb always have exceptions...}
  \small
  \begin{columns}[T]
    \begin{column}{0.5\linewidth}
      Here are some time-resolved data.  Clearly we cannot take more
      than one scan under any set of conditions because time marches on.        
    \end{column}
    \begin{column}{0.5\linewidth}
      The EXAFS data were taken at points in a rather large
      fluorescence imaging map.  We simply did not have enough time
      to measure more than a single scan and cover a large area.        
    \end{column}
  \end{columns}

  \medskip

  \begin{columns}[T]
    \begin{column}{0.5\linewidth}
      \includegraphics[width=\linewidth, height=3cm]{dummy.png}
    \end{column}
    \begin{column}{0.5\linewidth}
      \includegraphics[width=\linewidth, height=3cm]{dummy.png}
    \end{column}
  \end{columns}
  \begin{textblock*}{0.4\linewidth}(0pt,18.0\TPVertModule)%
    \tiny%
    S.R.\ Bare, et al., \textit{Characterizing industrial catalysts
      using in situ XAFS under identical conditions},
    Phys. Chem. Chem. Phys. (2010) \textbf{12}, pp\ 7702-7711
    \href{http://dx.doi.org/10.1039/B926621F}
    {\color{Blue4}\texttt{DOI:10.1039/B926621F}}
  \end{textblock*}
  \begin{textblock*}{0.4\linewidth}(11\TPHorizModule,18.0\TPVertModule)%
    \tiny%
    D.H.\ Phillips, et al., \textit{Deposition of Uranium Precipitates
      in Dolomitic Gravel Fill}, Environ. Sci. Technol. (2008)
    \textbf{42}:19, pp\ 7104–7110
    \href{http://dx.doi.org/10.1021/es8001579}
    {\color{Blue4}\texttt{DOI:10.1016/10.1021/es8001579}}
  \end{textblock*}
\end{frame}

\begin{frame}
  \frametitle{Statistical noise}
  What does it mean to say that data are dominated by statistical
  noise?
  \begin{enumerate}
  \item Your sample is well made
    \begin{itemize}
    \item homogeneous in the distribution of the absorber
    \item of an appropriate thickness
    \item not suffering from Bragg diffraction off of the sample or matrix
    \end{itemize}
  \item Your detectors are linear
    \begin{itemize}
    \item well constructed
    \item not saturated
    \item the entire signal chain is in a linear regime
    \end{itemize}
  \item The source and all optics are stable
  \end{enumerate}
  \begin{alertblock}{}
    If all of those conditions are met, the variance in your data will
    be statistical and subject to the CLT.
  \end{alertblock}
\end{frame}

\begin{frame}
  \frametitle{Making decisions with real data}
  Take a look at this

  \begin{center}
    Show a picture of crap data\\
    \includegraphics[width=0.5\linewidth, height=3cm]{dummy.png}
  \end{center}


  Three repititions will not be enough.  These data are simply not
  good enough.
\end{frame}

\section{Statistical analysis}

\begin{frame}
  \frametitle{An ensemble of data}
  \begin{columns}[T]
    \begin{column}{0.5\linewidth}
      one scan\\
      \includegraphics[width=\linewidth, height=3cm]{dummy.png}
    \end{column}
    \begin{column}{0.5\linewidth}
      all the scans overplotted\\
      \includegraphics[width=\linewidth, height=3cm]{dummy.png}
    \end{column}
  \end{columns}
\end{frame}

\begin{frame}
  \frametitle{The data merged}
  \begin{columns}[T]
    \begin{column}{0.5\linewidth}
      one scan + merge in k\\
      \includegraphics[width=\linewidth, height=3cm]{dummy.png}      
    \end{column}
    \begin{column}{0.5\linewidth}
      one scan + merge in R\\
      \includegraphics[width=\linewidth, height=3cm]{dummy.png}      
    \end{column}
  \end{columns}
\end{frame}

\begin{frame}
  \frametitle{Convergance to the mean}
  \begin{columns}[T]
    \begin{column}{0.5\linewidth}
      Overplot 1 4 9 25 all\\
      \includegraphics[width=\linewidth, height=3cm]{dummy.png}
    \end{column}
    \begin{column}{0.5\linewidth}
      \begin{tabular}[h]{cccc}
        scans & $\sqrt{N}$ & $\epsilon_k$ & ratio \\
        \hline\\
        1   & 1  & $x.xxx\times 10^{-3}$ & 1.0 \\
        4   & 2  & $x.xxx\times 10^{-3}$ & x.x \\
        16  & 4  & $x.xxx\times 10^{-4}$ & x.x \\
        36  & 6  & $x.xxx\times 10^{-4}$ & x.x \\
        64  & 8  & $x.xxx\times 10^{-4}$ & x.x \\
        100 & 10 & $x.xxx\times 10^{-4}$ & x.x \\
        142 & 11.9 & $x.xxx\times 10^{-4}$ & x.x
      \end{tabular}
    \end{column}
  \end{columns}
\end{frame}


\begin{frame}
  \frametitle{Data limitations}
  Here are successive samplings of 142 scans measured on a sample of
  Cr$_2$O$_3$.  The merge changes little after 16 scans.

  \bigskip

  \begin{columns}[T]
    \begin{column}{0.5\linewidth}
      \includegraphics[width=\linewidth]{images/cr2o3.png}
    \end{column}
    \begin{column}{0.5\linewidth}
      \small
      \begin{tabular}[h]{cccc}
        scans & $\sqrt{N}$ & $\epsilon_k$ & ratio \\
        \hline\\
        1   & 1  & $3.038\times 10^{-3}$ & 1.0 \\
        4   & 2  & $1.420\times 10^{-3}$ & 2.1 \\
        16  & 4  & $8.339\times 10^{-4}$ & 3.6 \\
        36  & 6  & $7.185\times 10^{-4}$ & 4.2 \\
        64  & 8  & $5.873\times 10^{-4}$ & 5.2 \\
        100 & 10 & $5.419\times 10^{-4}$ & 5.6 \\
        142 & 11.9 & $5.072\times 10^{-4}$ & 6.0
      \end{tabular}
    \end{column}
  \end{columns}

  \bigskip

  \begin{alertblock}{}
    \centering What's going on here?
  \end{alertblock}
\end{frame}

\begin{frame}
  \frametitle{Systematic uncertainty}
  More repititions only solves the problem of statistical noise.
  There is systematic error -- probably sample inhomogeneity -- in the
  Cr$_2$O$_3$ data at the level of $\epsilon_k \approx 5\times
  10^{-4}$.

  \bigskip

  Here are a couple of obvious examples of systematic problems.

  \begin{columns}[T]
    \begin{column}{0.5\linewidth}
      \centering Mono glitch\\
      \includegraphics[width=\linewidth, height=3cm]{dummy.png}      
    \end{column}
    \begin{column}{0.5\linewidth}
      \centering Samply inhomogeneity?\\
      \includegraphics[width=\linewidth, height=3cm]{dummy.png}      
    \end{column}
  \end{columns}

\end{frame}

\section{Conclusions}

\begin{frame}
  \frametitle{Blah}
  
\end{frame}


\end{document}



%%% Local Variables:
%%% TeX-parse-self: t
%%% TeX-auto-save: t
%%% TeX-auto-untabify: t
%%% TeX-PDF-mode: t
%%% End:
